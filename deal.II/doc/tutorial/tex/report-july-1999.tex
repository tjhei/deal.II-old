%% $Id$

\documentclass[a4paper]{article}
\usepackage[latin1]{inputenc}

\newcommand{\dealii}{{\em deal.II\/}}

\title{Report on the \dealii{} tutorial}
\author{Jan Schrage\footnote{Universit�t Heidelberg, Institut f�r theoretische Astrophysik \& SFB 359}}


\begin{document}
\maketitle

\begin{abstract}
We will describe the \dealii{} tutorial, its purpose
its state at the time of this writing, the standards employed
in writing it and the means to extend it.
\end{abstract}

\clearpage

\tableofcontents

\clearpage

\section{\dealii}
\label{sec:dealii}

\dealii{} is a C++ class library for the solution of systems 
of differential equations with finite element methods.

\section{Purpose of the tutorial}
\label{sec:purpose}
There are about one thousand pages of documentation of the \dealii{}
classes. However, there was not introductory documentation
nor instructions on the use of \dealii{} before the start of this
project. The purpose of this tutorial is to enable students or
other persons interested in \dealii{} to get an overview of the
methods employed and the use of the C++ classes. One should also
be able to use it as a quick reference to look up concepts of \dealii{}.
It should provide a background for a closer inspection
of the existing class documentation. Its purpose is not to replace
the class documentation but to complement it.

We have chosen to write the tutorial for viewing on the world wide web
in order to make it easily usable and
easily adaptable to the rapid changes in \dealii{}. 

\section{State of the tutorial}
\label{sec:state}

There are at the time of this writing two main parts of the tutorial:
\begin{itemize}
\item A general overview of the structure of a finite element program.
  This is the most crucial part laying the background for working with
  \dealii{}; it explains how and when to use what methods in a finite
  element program. This part is up--to--date.
\item The Laplace problem. This part gives a detailed explanation of
  the source code of a program solving the stationary Laplace problem 
  with \dealii{}. Due to the  rapid changes in \dealii{} the source
  code of this part is not compatible with the current
  version of \dealii{} any more.
\end{itemize}
Other parts can easily be added as needed following the steps 
in appendix~\ref{sec:steps}.


\section{Standards employed}
\label{sec:standards}

The \dealii{} tutorial makes use of HTML4 including frames and CSS1.
None of its pages contain dynamic content, in particular they do not
contain JavaScript\footnote{JavaScript is a registered trademark 
of Sun Microsystems, inc.}
and they do not require server--side preprocessing.

\subsection{HTML4 and CSS}
\label{sec:html4}

The tutorial makes use of HTML4\footnote{c.f. \cite{html4,htmlhelp4,niederst:webdesign}}
including frames
and Cascading Style Sheets (CSS) Level 1.
\footnote{c.f. \cite{css1,htmlhelpcss,niederst:webdesign}}
Even though
not all web browsers currently support CSS we expect this technology
to gain more widespread support in the near future. The browsers
currently at use in the IWR support a large part of the 
specifications of CSS1 and HTML4.

All the web pages has been validated against the HTML4 Document Type
Definition (DTD)\footnote{c.f. \cite{html4dtd}} and are guaranteed to be
syntactically correct. 


\subsection{Practical implications}
\label{sec:standards:implications}


\subsubsection{Usability by third parties}

\begin{list}{Frames}{}
\item[{\bf Frames:}] Frames are supported by most web browsers, even older versions.
  It is possible to view the tutorial without frames if a browser does not
  support them.
\item[{\bf HTML4:}] HTML4 to its full extent is supported by almost no browser; 
  Netscape Navigator 4 (NN4) and Microsoft Internet Explorer 4 (MSIE4)
  support a large
  subset of HTML4 and even older browsers can render most of it. We have
  taken care to use a subset of HTML4 mostly 
  compatible with older versions of HTML. Browsers that do not 
  support HTML4 will render small parts of the web pages incorrectly.
\item[{\bf CSS:}] Cascading Style Sheets Level 1 to their full extent are not 
  supported
  by any browser currently available. Again, NN4 and MSIE4 support a large 
  subset of CSS1. Browsers that do not support CSS1 will render the 
  web pages differently but still in a way that leaves them usable.
\end{list}
The browsers in use at the IWR render all the web pages correctly.

The use of CSS makes it possible to use different style sheets
for different media types, i.e. paper or audio for the 
blind. In the long and even medium run the usability of the tutorial
is thereby very much increased. Currently the tutorial uses
style sheets for screen, print and audio. The screen and print style
sheets are identical at the moment, the audio style sheet merely 
offers access to all the classes in use, as we are not aware of any
audio applications with the ability to parse style sheets.

\subsubsection{Writers}

The practical implications for writers are more eminent:
The use of CSS allows the separation of the page design from its content.
Any writer extending the tutorial only needs to include the
appropriate CSSs in order to produce a web page with the same
design as the rest of the tutorial suitable for use with 
different media. 


\section{Use of templates}
\label{sec:templates}

In order to facilitate the extension of the tutorial there
are a number of template files for recurring elements.
Anybody who wishes to use such an element only need to include
and adapt the appropriate template. Where replacements
are needed in a template they are prefixed with ``INSERT\_''
to ease the adaption to any specific case.

A more detailed explanation is provided in appendix~\ref{sec:steps}.


\section{A question of style}
\label{sec:style}

Within the \dealii{} tutorial style and content are (with very few 
exceptions) strictly separated. This makes the design easily changeable
and adaptable to different needs. It also helps to keep the design 
and style consistent throughout the tutorial. 
We have used abstract HTML markup, defining classes for different
element types. The visual (or aural) markup of each class for different media
is defined in the style sheets. 

We have taken care to ensure easy and consistent means of navigation
throughout the tutorial; we have also tried to suit the needs of the
users and not those of the programmers of \dealii{}.

In the bibliography we list a few references on good HTML style, like
\cite{galactus:style,fleming:webnavigation}.

\section{Feedback so far}
\label{sec:feedback}

Feedback on the tutorial so far has been positive.
The methods of navigation and the general look and feel of the
\dealii{} tutorial have met with agreement.

The most eminent wish by students was that more examples in the 
way of working programs should be provided to complement the
tutorial. At the time of this writing the examples provided
are stripped of all the program code that is not strictly necessary 
to develop an understanding of the principles. In particular,
there are usually no compilable and runnable programs provided. 

\clearpage

\appendix

\section{Directory structure}
\label{sec:directories}

All the relevant files and directories are contained in the directory 
{\tt doc/tutorial} in the \dealii{} directory structure.
At the moment the following subdirectories and files exist on the top
level:
\par
\vspace{0.5cm}
\begin{tabular}[H]{l|r|p{6cm}}
Name & Revision number & Description \\
\hline
index.html & 1.10 & tutorial entry page \\
screen.css & 1.1 & CSS1 style sheet, media: screen \\
print.css & 1.1 & CSS1 style sheet, media: print media \\
audio.css & 1.1 & CSS1 style sheet, media: aural media \\
templates/ & --- & template directory \\
figures/ & --- & figures in {\tt eps} or {\tt fig} format\\
images/ & --- & figures in {\tt gif} format for the web pages\\
glossary/ & --- & glossary explaining the terms used \\
scripts/ & --- & scripts for maintenance and validation\\
tex/ & --- & reports \\
00.fe\_structure/ & --- & Chapter 0: Structure of a finite element program \\
01.laplace/ & --- & Chapter 1: The Laplace problem\\
\end{tabular}

\section{CSS1--classes in use}
\label{sec:css-classes}

Below we show the CSS1--classes currently in use and the 
structural elements to which they should be applied, i.e.
which elements should have the attribute ``class=...'' set.
\par
\vspace{0.5cm}
\begin{tabular}[H]{l|p{6cm}}
Class & Corresponding Elements \\
\hline
figure & figures, subtitles to figures \\
pagetoc & small table of contents for only one (long) page \\
chapter\_title & title of a chapter, used in the appropriate templates\\
parhead & paragraph heading \\
example & examples, used in the appropriate template \\
navbar & navigation bar at the page bottom\\
\end{tabular}
\par
\vspace{0.5cm}
In addition to this we have defined some properties common to 
all elements of a particular type, like background colour for the
page body etc.


\section{Extending the tutorial -- step by step}
\label{sec:steps}

\begin{enumerate}
\item Create a directory with the number and name of the new
  chapter like ``02.newchapter''.
\item Copy the following files from the directory ``templates''
  into your new directory: 
  {\tt
  \begin{itemize}
  \item toc.html
  \item index.html
  \item navbar.html
  \item title.html
  \item logo.html
  \end{itemize}
  }
\item Adapt these files according to your needs. {\tt toc.html} contains
  the long table of contents with explanations, {\tt navbar.html} the 
  brief table of contents that can be seen on the left, {\tt index.html}
  puts the layout of the frameset together and {\tt title.html} contains
  the large chapter title on top. Take care to replace all the ``INSERT\_''
  tags with sensible content.
\item Create your web pages using {\tt body.html} from the template
  directory. Again replace all ``INSERT\_'' tags with sensible content.
\item For code examples within a web page 
  use {\tt example.html} from the template
  directory. This particular template can be inserted into any
  web page.
\item Create a link to your {\tt index.html} from the tutorial's 
  entry page.
\end{enumerate}

\clearpage
\bibliographystyle{alpha}
\bibliography{tutorial}

\end{document}
