\batchmode


\documentclass[a4paper,11pt]{article}
\RequirePackage{ifthen}


\NeedsTeXFormat{LaTeX2e}
\usepackage{exscale} 
\usepackage[dvips]{graphicx}
\usepackage{shortvrb}  
\usepackage{amsmath}   
\usepackage{amssymb}   
\usepackage{amsfonts}  
\usepackage{graphicx}	

\addtolength{\topmargin}{-35pt}\addtolength{\headsep}{-2pt}\addtolength{\topskip}{-5pt}\addtolength{\oddsidemargin}{-1.5cm}\addtolength{\evensidemargin}{-1.5cm}\addtolength{\textheight}{45pt}%% less white space at bottom of page

\addtolength{\textwidth}{4cm}%% larger columns


%
\providecommand{\vect}[1]{\underline{#1}}%% vectors%
\providecommand{\matr}[1]{\mathbf{#1}}%% matrices%
\providecommand{\ofx}{(\underline{x})}%
\providecommand{\oftx}{(t,\underline{x})}%
\providecommand{\R}{\mathbb{R}}%% number sets%
\providecommand{\Z}{\mathbb{Z}}%
\providecommand{\C}{\mathbb{C}}%
\providecommand{\N}{\mathbb{N}}%
\providecommand{\inR}[1]{\in \mathbb{R}^{#1}}%
\providecommand{\EE}[1]{\mathbb{E}\,#1}%% mathematical expectation%
\providecommand{\PP}[1]{\mathbb{P}\,#1}%% mathematical probability%
\providecommand{\Or}[2]{\mathcal{O}(#1^#2)}%% order%
\providecommand{\eye}[1]{\,\mathbb{I}_{#1}\,}%% identity matrix%
\providecommand{\Laplace}{\Delta}%% Laplace operator%
\providecommand{\Grad}{\underline{\nabla}}%% Gradient operator%
\providecommand{\ond}[1]{\in \partial#1}%% on (physical domain .. ) boundary%
\providecommand{\etime}{\tau^D_{\underline{x}}}%
\providecommand{\twovec}[2]{\left(\begin{array}{c}#1\\#2\end{array}\right)}%
\providecommand{\threevec}[3]{\left(\begin{array}{c}#1\\#2\\#3\end{array}\right)}

\parindent 0pt

%
\providecommand{\Title}[1]{\title{\Large{#1}}  \author{\small{Anna Schneebeli, \today}}\date{}}%
\providecommand{\Abstract}[1]{\noindent \small \textbf{Abstract:} #1}%
\providecommand{\Section}[1]{\section{\large{#1}}}%
\providecommand{\SectionS}[1]{\section*{\large{#1}}}%
\providecommand{\Subsection}[1]{\subsection{\normalsize{#1}}}%
\providecommand{\SubsectionS}[1]{\subsection*{\normalsize{#1}}}%
\providecommand{\Subsubsection}[1]{\subsubsection{\normalsize{#1}}}%
\providecommand{\SubsubsectionS}[1]{\subsubsection*{\normalsize{#1}}}
\newtheorem{remark}{\mdseries{\textsc{Remark}}}\newtheorem{conjecture}{\mdseries{\textsc{Conjecture}}}
\bibliographystyle{abbrv}

%
\providecommand{\proof}{\mdseries{\textsc{Proof. }}}%
\providecommand{\qed}{\begin{flushright} $\square$\  \end{flushright}}
\newtheorem{definition}{\mdseries{\textsc{Definition}}}\newtheorem{theorem}{\mdseries{\textsc{Theorem}}}\newtheorem{prop}{\mdseries{\textsc{Proposition}}}\newtheorem{example}{\mdseries{\textsc{Example}}}\newtheorem{corollary}{\mdseries{\textsc{Corollary}}}\newtheorem{lemma}{\mdseries{\textsc{Lemma}}}\newtheorem{convention}{\mdseries{\textsc{Convention}}}
\title{\Large {An $H(\mathop{\rm curl};\Omega )$-conforming FEM: \\
		N\'ed\'elec's elements of first type}}  \author{\small{Anna Schneebeli, \today}}\date{}


\usepackage[dvips]{color}


\pagecolor[gray]{.7}

\usepackage[latin1]{inputenc}



\makeatletter

\makeatletter
\count@=\the\catcode`\_ \catcode`\_=8 
\newenvironment{tex2html_wrap}{}{}%
\catcode`\<=12\catcode`\_=\count@
\newcommand{\providedcommand}[1]{\expandafter\providecommand\csname #1\endcsname}%
\newcommand{\renewedcommand}[1]{\expandafter\providecommand\csname #1\endcsname{}%
  \expandafter\renewcommand\csname #1\endcsname}%
\newcommand{\newedenvironment}[1]{\newenvironment{#1}{}{}\renewenvironment{#1}}%
\let\newedcommand\renewedcommand
\let\renewedenvironment\newedenvironment
\makeatother
\let\mathon=$
\let\mathoff=$
\ifx\AtBeginDocument\undefined \newcommand{\AtBeginDocument}[1]{}\fi
\newbox\sizebox
\setlength{\hoffset}{0pt}\setlength{\voffset}{0pt}
\addtolength{\textheight}{\footskip}\setlength{\footskip}{0pt}
\addtolength{\textheight}{\topmargin}\setlength{\topmargin}{0pt}
\addtolength{\textheight}{\headheight}\setlength{\headheight}{0pt}
\addtolength{\textheight}{\headsep}\setlength{\headsep}{0pt}
\setlength{\textwidth}{349pt}
\newwrite\lthtmlwrite
\makeatletter
\let\realnormalsize=\normalsize
\global\topskip=2sp
\def\preveqno{}\let\real@float=\@float \let\realend@float=\end@float
\def\@float{\let\@savefreelist\@freelist\real@float}
\def\liih@math{\ifmmode$\else\bad@math\fi}
\def\end@float{\realend@float\global\let\@freelist\@savefreelist}
\let\real@dbflt=\@dbflt \let\end@dblfloat=\end@float
\let\@largefloatcheck=\relax
\let\if@boxedmulticols=\iftrue
\def\@dbflt{\let\@savefreelist\@freelist\real@dbflt}
\def\adjustnormalsize{\def\normalsize{\mathsurround=0pt \realnormalsize
 \parindent=0pt\abovedisplayskip=0pt\belowdisplayskip=0pt}%
 \def\phantompar{\csname par\endcsname}\normalsize}%
\def\lthtmltypeout#1{{\let\protect\string \immediate\write\lthtmlwrite{#1}}}%
\newcommand\lthtmlhboxmathA{\adjustnormalsize\setbox\sizebox=\hbox\bgroup\kern.05em }%
\newcommand\lthtmlhboxmathB{\adjustnormalsize\setbox\sizebox=\hbox to\hsize\bgroup\hfill }%
\newcommand\lthtmlvboxmathA{\adjustnormalsize\setbox\sizebox=\vbox\bgroup %
 \let\ifinner=\iffalse \let\)\liih@math }%
\newcommand\lthtmlboxmathZ{\@next\next\@currlist{}{\def\next{\voidb@x}}%
 \expandafter\box\next\egroup}%
\newcommand\lthtmlmathtype[1]{\gdef\lthtmlmathenv{#1}}%
\newcommand\lthtmllogmath{\lthtmltypeout{l2hSize %
:\lthtmlmathenv:\the\ht\sizebox::\the\dp\sizebox::\the\wd\sizebox.\preveqno}}%
\newcommand\lthtmlfigureA[1]{\let\@savefreelist\@freelist
       \lthtmlmathtype{#1}\lthtmlvboxmathA}%
\newcommand\lthtmlpictureA{\bgroup\catcode`\_=8 \lthtmlpictureB}%
\newcommand\lthtmlpictureB[1]{\lthtmlmathtype{#1}\egroup
       \let\@savefreelist\@freelist \lthtmlhboxmathB}%
\newcommand\lthtmlpictureZ[1]{\hfill\lthtmlfigureZ}%
\newcommand\lthtmlfigureZ{\lthtmlboxmathZ\lthtmllogmath\copy\sizebox
       \global\let\@freelist\@savefreelist}%
\newcommand\lthtmldisplayA{\bgroup\catcode`\_=8 \lthtmldisplayAi}%
\newcommand\lthtmldisplayAi[1]{\lthtmlmathtype{#1}\egroup\lthtmlvboxmathA}%
\newcommand\lthtmldisplayB[1]{\edef\preveqno{(\theequation)}%
  \lthtmldisplayA{#1}\let\@eqnnum\relax}%
\newcommand\lthtmldisplayZ{\lthtmlboxmathZ\lthtmllogmath\lthtmlsetmath}%
\newcommand\lthtmlinlinemathA{\bgroup\catcode`\_=8 \lthtmlinlinemathB}
\newcommand\lthtmlinlinemathB[1]{\lthtmlmathtype{#1}\egroup\lthtmlhboxmathA
  \vrule height1.5ex width0pt }%
\newcommand\lthtmlinlineA{\bgroup\catcode`\_=8 \lthtmlinlineB}%
\newcommand\lthtmlinlineB[1]{\lthtmlmathtype{#1}\egroup\lthtmlhboxmathA}%
\newcommand\lthtmlinlineZ{\egroup\expandafter\ifdim\dp\sizebox>0pt %
  \expandafter\centerinlinemath\fi\lthtmllogmath\lthtmlsetinline}
\newcommand\lthtmlinlinemathZ{\egroup\expandafter\ifdim\dp\sizebox>0pt %
  \expandafter\centerinlinemath\fi\lthtmllogmath\lthtmlsetmath}
\newcommand\lthtmlindisplaymathZ{\egroup %
  \centerinlinemath\lthtmllogmath\lthtmlsetmath}
\def\lthtmlsetinline{\hbox{\vrule width.1em \vtop{\vbox{%
  \kern.1em\copy\sizebox}\ifdim\dp\sizebox>0pt\kern.1em\else\kern.3pt\fi
  \ifdim\hsize>\wd\sizebox \hrule depth1pt\fi}}}
\def\lthtmlsetmath{\hbox{\vrule width.1em\kern-.05em\vtop{\vbox{%
  \kern.1em\kern0.8 pt\hbox{\hglue.17em\copy\sizebox\hglue0.8 pt}}\kern.3pt%
  \ifdim\dp\sizebox>0pt\kern.1em\fi \kern0.8 pt%
  \ifdim\hsize>\wd\sizebox \hrule depth1pt\fi}}}
\def\centerinlinemath{%
  \dimen1=\ifdim\ht\sizebox<\dp\sizebox \dp\sizebox\else\ht\sizebox\fi
  \advance\dimen1by.5pt \vrule width0pt height\dimen1 depth\dimen1 
 \dp\sizebox=\dimen1\ht\sizebox=\dimen1\relax}

\def\lthtmlcheckvsize{\ifdim\ht\sizebox<\vsize 
  \ifdim\wd\sizebox<\hsize\expandafter\hfill\fi \expandafter\vfill
  \else\expandafter\vss\fi}%
\providecommand{\selectlanguage}[1]{}%
\makeatletter \tracingstats = 1 
\providecommand{\Eta}{\textrm{H}}
\providecommand{\Mu}{\textrm{M}}
\providecommand{\Alpha}{\textrm{A}}
\providecommand{\Iota}{\textrm{J}}
\providecommand{\Nu}{\textrm{N}}
\providecommand{\Omicron}{\textrm{O}}
\providecommand{\omicron}{\textrm{o}}
\providecommand{\Chi}{\textrm{X}}
\providecommand{\Beta}{\textrm{B}}
\providecommand{\Kappa}{\textrm{K}}
\providecommand{\Tau}{\textrm{T}}
\providecommand{\Epsilon}{\textrm{E}}
\providecommand{\Zeta}{\textrm{Z}}
\providecommand{\Rho}{\textrm{R}}


\begin{document}
\pagestyle{empty}\thispagestyle{empty}\lthtmltypeout{}%
\lthtmltypeout{latex2htmlLength hsize=\the\hsize}\lthtmltypeout{}%
\lthtmltypeout{latex2htmlLength vsize=\the\vsize}\lthtmltypeout{}%
\lthtmltypeout{latex2htmlLength hoffset=\the\hoffset}\lthtmltypeout{}%
\lthtmltypeout{latex2htmlLength voffset=\the\voffset}\lthtmltypeout{}%
\lthtmltypeout{latex2htmlLength topmargin=\the\topmargin}\lthtmltypeout{}%
\lthtmltypeout{latex2htmlLength topskip=\the\topskip}\lthtmltypeout{}%
\lthtmltypeout{latex2htmlLength headheight=\the\headheight}\lthtmltypeout{}%
\lthtmltypeout{latex2htmlLength headsep=\the\headsep}\lthtmltypeout{}%
\lthtmltypeout{latex2htmlLength parskip=\the\parskip}\lthtmltypeout{}%
\lthtmltypeout{latex2htmlLength oddsidemargin=\the\oddsidemargin}\lthtmltypeout{}%
\makeatletter
\if@twoside\lthtmltypeout{latex2htmlLength evensidemargin=\the\evensidemargin}%
\else\lthtmltypeout{latex2htmlLength evensidemargin=\the\oddsidemargin}\fi%
\lthtmltypeout{}%
\makeatother
\setcounter{page}{1}
\onecolumn

% !!! IMAGES START HERE !!!

{\newpage\clearpage
\lthtmlinlinemathA{tex2html_wrap_inline5220}%
$ H(\mathop {\rm curl};\Omega )$%
\lthtmlinlinemathZ
\lthtmlcheckvsize\clearpage}

{\newpage\clearpage
\lthtmlinlinemathA{tex2html_wrap_inline5236}%
$ L^2(\Omega )$%
\lthtmlinlinemathZ
\lthtmlcheckvsize\clearpage}

{\newpage\clearpage
\lthtmlinlinemathA{tex2html_wrap_inline5250}%
$ H(\mathop {\rm curl};(\Omega ))$%
\lthtmlinlinemathZ
\lthtmlcheckvsize\clearpage}

\stepcounter{section}
{\newpage\clearpage
\lthtmlinlinemathA{tex2html_wrap_inline5451}%
$ \Omega \in \mathbb{R}^d$%
\lthtmlinlinemathZ
\lthtmlcheckvsize\clearpage}

{\newpage\clearpage
\lthtmlinlinemathA{tex2html_wrap_inline5453}%
$ d=2,3$%
\lthtmlinlinemathZ
\lthtmlcheckvsize\clearpage}

{\newpage\clearpage
\lthtmlinlinemathA{tex2html_wrap_indisplay5455}%
$\displaystyle \mathop{\rm curl}\mathop{\rm curl}\underline u + c(x) \underline u  = \underline f \quad \mathrm{in} \quad \Omega \,,$%
\lthtmlindisplaymathZ
\lthtmlcheckvsize\clearpage}

{\newpage\clearpage
\lthtmlinlinemathA{tex2html_wrap_inline5457}%
$ \underline f \in L^2(\Omega )^d$%
\lthtmlinlinemathZ
\lthtmlcheckvsize\clearpage}

{\newpage\clearpage
\lthtmlinlinemathA{tex2html_wrap_indisplay5459}%
$\displaystyle \underline u \wedge \underline n  = 0$%
\lthtmlindisplaymathZ
\lthtmlcheckvsize\clearpage}

{\newpage\clearpage
\lthtmlinlinemathA{tex2html_wrap_inline5461}%
$ \partial \Omega $%
\lthtmlinlinemathZ
\lthtmlcheckvsize\clearpage}

{\newpage\clearpage
\lthtmlinlinemathA{tex2html_wrap_inline5463}%
$ \Omega $%
\lthtmlinlinemathZ
\lthtmlcheckvsize\clearpage}

{\newpage\clearpage
\lthtmlinlinemathA{tex2html_wrap_inline5465}%
$ c(x)$%
\lthtmlinlinemathZ
\lthtmlcheckvsize\clearpage}

\stepcounter{subsection}
{\newpage\clearpage
\lthtmlinlinemathA{tex2html_wrap_inline5469}%
$ \underline t$%
\lthtmlinlinemathZ
\lthtmlcheckvsize\clearpage}

{\newpage\clearpage
\lthtmlinlinemathA{tex2html_wrap_inline5471}%
$ d=2$%
\lthtmlinlinemathZ
\lthtmlcheckvsize\clearpage}

{\newpage\clearpage
\lthtmlinlinemathA{tex2html_wrap_inline5473}%
$ \underline v = \left(\begin{array}{c} v_1(x,y) \\   v_2(x,y)\end{array} \right) \in [\mathcal{D}(\overline{\Omega })]^2$%
\lthtmlinlinemathZ
\lthtmlcheckvsize\clearpage}

{\newpage\clearpage
\lthtmlinlinemathA{tex2html_wrap_inline5475}%
$ \varphi \in \mathcal{D}(\overline{\Omega })$%
\lthtmlinlinemathZ
\lthtmlcheckvsize\clearpage}

{\newpage\clearpage
\lthtmlinlinemathA{tex2html_wrap_indisplay5477}%
$\displaystyle \mathop{\rm curl}\underline v := \partial _x v_2 - \partial _y v_1   \quad \mathrm{and} \quad \mathop{\underline{\rm curl}}\varphi := \left(\begin{array}{c} \partial _y\varphi \\  -\partial _x\varphi \end{array}
\right) \,.
$%
\lthtmlindisplaymathZ
\lthtmlcheckvsize\clearpage}

{\newpage\clearpage
\lthtmlinlinemathA{tex2html_wrap_inline5479}%
$ \mathop{\rm curl}\mathop{\rm curl}$%
\lthtmlinlinemathZ
\lthtmlcheckvsize\clearpage}

{\newpage\clearpage
\lthtmlinlinemathA{tex2html_wrap_inline5481}%
$ \mathop{\underline{\rm curl}}\mathop{\rm curl}$%
\lthtmlinlinemathZ
\lthtmlcheckvsize\clearpage}

{\newpage\clearpage
\lthtmlinlinemathA{tex2html_wrap_inline5484}%
$ \mathop{\rm curl}$%
\lthtmlinlinemathZ
\lthtmlcheckvsize\clearpage}

{\newpage\clearpage
\lthtmlinlinemathA{tex2html_wrap_inline5486}%
$ \underline v$%
\lthtmlinlinemathZ
\lthtmlcheckvsize\clearpage}

{\newpage\clearpage
\lthtmlinlinemathA{tex2html_wrap_inline5488}%
$ \mathop{\underline{\rm curl}}$%
\lthtmlinlinemathZ
\lthtmlcheckvsize\clearpage}

{\newpage\clearpage
\lthtmlinlinemathA{tex2html_wrap_inline5490}%
$ \varphi $%
\lthtmlinlinemathZ
\lthtmlcheckvsize\clearpage}

{\newpage\clearpage
\lthtmlinlinemathA{tex2html_wrap_indisplay5492}%
$\displaystyle \boldsymbol{R} = \left(\begin{array}{cc}
0 & 1 \\ 
-1 & 0
\end{array}\right) \,,
$%
\lthtmlindisplaymathZ
\lthtmlcheckvsize\clearpage}

{\newpage\clearpage
\lthtmlinlinemathA{tex2html_wrap_indisplay5494}%
$\displaystyle \mathop{\rm curl}\underline v = \mathrm{div} \left(\boldsymbol{R} \underline v\right) 
$%
\lthtmlindisplaymathZ
\lthtmlcheckvsize\clearpage}

{\newpage\clearpage
\lthtmlinlinemathA{tex2html_wrap_indisplay5496}%
$\displaystyle \mathop{\underline{\rm curl}}\varphi = \boldsymbol{R} \nabla\varphi \,.
$%
\lthtmlindisplaymathZ
\lthtmlcheckvsize\clearpage}

{\newpage\clearpage
\lthtmlinlinemathA{tex2html_wrap_inline5500}%
$ \underline t = \boldsymbol{R}^T\underline n$%
\lthtmlinlinemathZ
\lthtmlcheckvsize\clearpage}

{\newpage\clearpage
\lthtmlinlinemathA{tex2html_wrap_inline5504}%
$ d=3$%
\lthtmlinlinemathZ
\lthtmlcheckvsize\clearpage}

{\newpage\clearpage
\lthtmlinlinemathA{tex2html_wrap_inline5506}%
$ \underline v \in [\mathcal{D}(\overline{\Omega })]^3$%
\lthtmlinlinemathZ
\lthtmlcheckvsize\clearpage}

{\newpage\clearpage
\lthtmlinlinemathA{tex2html_wrap_indisplay5508}%
$\displaystyle \mathop{\rm curl}\underline v := \nabla \wedge \underline v := \left(\begin{array}{c} 
\partial _y v_3 - \partial _z v_2 \\ 
\partial _z v_1 - \partial _x v_3 \\ 						 
\partial _x v_2 - \partial _y v_1						 
\end{array} \right)
$%
\lthtmlindisplaymathZ
\lthtmlcheckvsize\clearpage}

{\newpage\clearpage
\lthtmlinlinemathA{tex2html_wrap_inline5513}%
$ \tilde{d}=1$%
\lthtmlinlinemathZ
\lthtmlcheckvsize\clearpage}

{\newpage\clearpage
\lthtmlinlinemathA{tex2html_wrap_inline5517}%
$ \tilde{d}=3$%
\lthtmlinlinemathZ
\lthtmlcheckvsize\clearpage}

{\newpage\clearpage
\lthtmlinlinemathA{tex2html_wrap_indisplay5521}%
$\displaystyle H(\mathop{\rm curl}; \Omega ) := \{ \underline v \in [L^2(\Omega )]^d:  \mathop{\rm curl}\underline v \in [L^2(\Omega )]^{\tilde{d}} \}
$%
\lthtmlindisplaymathZ
\lthtmlcheckvsize\clearpage}

{\newpage\clearpage
\lthtmlinlinemathA{tex2html_wrap_indisplay5525}%
$\displaystyle (\underline v, \underline u)_{H(\mathop{\rm curl};\Omega )} := (\underline v, \underline u)_{L^2(\Omega )} + (\mathop{\rm curl}\underline v, \mathop{\rm curl}\underline u)_{L^2(\Omega )}
$%
\lthtmlindisplaymathZ
\lthtmlcheckvsize\clearpage}

\stepcounter{subsection}
{\newpage\clearpage
\lthtmlinlinemathA{tex2html_wrap_inline5539}%
$ [\mathcal{D}(\overline{\Omega })]^d$%
\lthtmlinlinemathZ
\lthtmlcheckvsize\clearpage}

{\newpage\clearpage
\lthtmlinlinemathA{tex2html_wrap_inline5546}%
$ \underline u$%
\lthtmlinlinemathZ
\lthtmlcheckvsize\clearpage}

{\newpage\clearpage
\lthtmlinlinemathA{tex2html_wrap_inline5548}%
$ [H(\mathop{\rm curl};\Omega )]^2$%
\lthtmlinlinemathZ
\lthtmlcheckvsize\clearpage}

{\newpage\clearpage
\lthtmlinlinemathA{tex2html_wrap_inline5552}%
$ H^1(\Omega )$%
\lthtmlinlinemathZ
\lthtmlcheckvsize\clearpage}

{\newpage\clearpage
\lthtmlinlinemathA{tex2html_wrap_indisplay5554}%
$\displaystyle \int_{\Omega } \mathop{\rm curl}\underline u \,\,\varphi \,dx  = \int_{\Omega } \underline u \cdot \mathop{\underline{\rm curl}}\varphi \,dx  + \int_{\partial \Omega } (\underline u\cdot \underline t) \, \varphi \,ds \:,
$%
\lthtmlindisplaymathZ
\lthtmlcheckvsize\clearpage}

{\newpage\clearpage
\lthtmlinlinemathA{tex2html_wrap_inline5558}%
$ [H(\mathop{\rm curl};\Omega )]^3$%
\lthtmlinlinemathZ
\lthtmlcheckvsize\clearpage}

{\newpage\clearpage
\lthtmlinlinemathA{tex2html_wrap_inline5562}%
$ [H^1(\Omega )]^3$%
\lthtmlinlinemathZ
\lthtmlcheckvsize\clearpage}

{\newpage\clearpage
\lthtmlinlinemathA{tex2html_wrap_indisplay5564}%
$\displaystyle \int_{\Omega } \underline v \cdot \mathop{\rm curl}\underline u \,dx  = \int_{\Omega } \underline u \cdot \mathop{\rm curl}\underline v \,dx  + \int_{\partial \Omega } (\underline v\wedge \underline n) \cdot \underline u\,ds \:,
$%
\lthtmlindisplaymathZ
\lthtmlcheckvsize\clearpage}

{\newpage\clearpage
\lthtmlinlinemathA{tex2html_wrap_inline5566}%
$ [H^{-\frac{1}{2}}(\partial \Omega )]^{\tilde{d}} \times H^{\frac{1}{2}}(\partial \Omega )$%
\lthtmlinlinemathZ
\lthtmlcheckvsize\clearpage}

{\newpage\clearpage
\lthtmlinlinemathA{tex2html_wrap_indisplay5568}%
$\displaystyle \mathrm{div}\, (\underline u \wedge \underline v) = \underline v \cdot \mathop{\rm curl}\underline u - \underline u \cdot \mathop{\rm curl}\underline v 
$%
\lthtmlindisplaymathZ
\lthtmlcheckvsize\clearpage}

{\newpage\clearpage
\lthtmlinlinemathA{tex2html_wrap_inline5570}%
$ (\underline a\wedge\underline b)\cdot \underline c$%
\lthtmlinlinemathZ
\lthtmlcheckvsize\clearpage}

{\newpage\clearpage
\lthtmlinlinemathA{tex2html_wrap_indisplay5572}%
$\displaystyle \int_{\Omega } \underline v\cdot \mathop{\rm curl}\underline u - \underline u\cdot \mathop{\rm curl}\underline v  \, dx = \int_{\Omega } \mathrm{div}\, (\underline u \wedge \underline v) \, dx
= \int_{\partial \Omega } (\underline u \wedge \underline v)\cdot \underline n \,ds = \int_{\partial \Omega } (\underline v \wedge \underline n)\cdot \underline u \,ds \,\,.
$%
\lthtmlindisplaymathZ
\lthtmlcheckvsize\clearpage}

{\newpage\clearpage
\lthtmlinlinemathA{tex2html_wrap_inline5574}%
$ H(\mathop{\rm curl})$%
\lthtmlinlinemathZ
\lthtmlcheckvsize\clearpage}

{\newpage\clearpage
\lthtmlinlinemathA{tex2html_wrap_inline5581}%
$ \underline n$%
\lthtmlinlinemathZ
\lthtmlcheckvsize\clearpage}

{\newpage\clearpage
\lthtmlinlinemathA{tex2html_wrap_indisplay5591}%
$\displaystyle \gamma: \quad \underline v \mapsto \gamma(\underline v) \cdot \underline t
$%
\lthtmlindisplaymathZ
\lthtmlcheckvsize\clearpage}

{\newpage\clearpage
\lthtmlinlinemathA{tex2html_wrap_indisplay5595}%
$\displaystyle \gamma: \quad \underline v \mapsto \gamma(\underline v) \wedge \underline n
$%
\lthtmlindisplaymathZ
\lthtmlcheckvsize\clearpage}

{\newpage\clearpage
\lthtmlinlinemathA{tex2html_wrap_inline5599}%
$ [H^{-\frac{1}{2}}(\partial \Omega )]^{\tilde{d}}$%
\lthtmlinlinemathZ
\lthtmlcheckvsize\clearpage}

{\newpage\clearpage
\lthtmlinlinemathA{tex2html_wrap_inline5603}%
$ [H^1(\Omega )]^{\tilde{d}}$%
\lthtmlinlinemathZ
\lthtmlcheckvsize\clearpage}

{\newpage\clearpage
\lthtmlinlinemathA{tex2html_wrap_inline5605}%
$ [H^{\frac{1}{2}}(\partial \Omega )]^{\tilde{d}}$%
\lthtmlinlinemathZ
\lthtmlcheckvsize\clearpage}

{\newpage\clearpage
\lthtmlinlinemathA{tex2html_wrap_indisplay5610}%
$\displaystyle H_0(\mathop{\rm curl};\Omega ) := \left\{ \underline v \in H(\mathop{\rm curl};\Omega ): \quad \underline v\wedge \underline n = 0 \:\:\mathrm{on}\:\: \partial \Omega \right\}
$%
\lthtmlindisplaymathZ
\lthtmlcheckvsize\clearpage}

{\newpage\clearpage
\lthtmlinlinemathA{tex2html_wrap_inline5615}%
$ [\mathcal{D}(\Omega )]^d$%
\lthtmlinlinemathZ
\lthtmlcheckvsize\clearpage}

{\newpage\clearpage
\lthtmlinlinemathA{tex2html_wrap_inline5617}%
$ H_0(\mathop{\rm curl};\Omega )$%
\lthtmlinlinemathZ
\lthtmlcheckvsize\clearpage}

{\newpage\clearpage
\lthtmlinlinemathA{tex2html_wrap_inline5622}%
$ K_-$%
\lthtmlinlinemathZ
\lthtmlcheckvsize\clearpage}

{\newpage\clearpage
\lthtmlinlinemathA{tex2html_wrap_inline5624}%
$ K_+$%
\lthtmlinlinemathZ
\lthtmlcheckvsize\clearpage}

{\newpage\clearpage
\lthtmlinlinemathA{tex2html_wrap_inline5626}%
$ \mathbb{R}^d$%
\lthtmlinlinemathZ
\lthtmlcheckvsize\clearpage}

{\newpage\clearpage
\lthtmlinlinemathA{tex2html_wrap_inline5628}%
$ e = \partial K_-\cap\partial K_+ \neq \emptyset$%
\lthtmlinlinemathZ
\lthtmlcheckvsize\clearpage}

{\newpage\clearpage
\lthtmlinlinemathA{tex2html_wrap_inline5630}%
$ \Omega = \partial K_-\cup\partial K_+$%
\lthtmlinlinemathZ
\lthtmlcheckvsize\clearpage}

{\newpage\clearpage
\lthtmlinlinemathA{tex2html_wrap_inline5632}%
$ v$%
\lthtmlinlinemathZ
\lthtmlcheckvsize\clearpage}

{\newpage\clearpage
\lthtmlinlinemathA{tex2html_wrap_inline5636}%
$ v_-$%
\lthtmlinlinemathZ
\lthtmlcheckvsize\clearpage}

{\newpage\clearpage
\lthtmlinlinemathA{tex2html_wrap_inline5642}%
$ H(\mathop{\rm curl}; K_-)$%
\lthtmlinlinemathZ
\lthtmlcheckvsize\clearpage}

{\newpage\clearpage
\lthtmlinlinemathA{tex2html_wrap_inline5644}%
$ v_+$%
\lthtmlinlinemathZ
\lthtmlcheckvsize\clearpage}

{\newpage\clearpage
\lthtmlinlinemathA{tex2html_wrap_inline5650}%
$ H(\mathop{\rm curl}; K_+)$%
\lthtmlinlinemathZ
\lthtmlcheckvsize\clearpage}

{\newpage\clearpage
\lthtmlinlinemathA{tex2html_wrap_inline5652}%
$ e$%
\lthtmlinlinemathZ
\lthtmlcheckvsize\clearpage}

{\newpage\clearpage
\lthtmlinlinemathA{tex2html_wrap_inline5654}%
$ v_-\wedge n_- + v_+\wedge n_+ = 0$%
\lthtmlinlinemathZ
\lthtmlcheckvsize\clearpage}

{\newpage\clearpage
\lthtmlinlinemathA{tex2html_wrap_inline5658}%
$ H^{\frac{1}{2}}_{00}(e)$%
\lthtmlinlinemathZ
\lthtmlcheckvsize\clearpage}

{\newpage\clearpage
\lthtmlinlinemathA{tex2html_wrap_inline5662}%
$ H^{\frac{1}{2}}(\partial \Omega )$%
\lthtmlinlinemathZ
\lthtmlcheckvsize\clearpage}

{\newpage\clearpage
\lthtmlinlinemathA{tex2html_wrap_inline5674}%
$ H^{-\frac{1}{2}}(e)$%
\lthtmlinlinemathZ
\lthtmlcheckvsize\clearpage}

{\newpage\clearpage
\lthtmlinlinemathA{tex2html_wrap_inline5676}%
$ \underline v_-$%
\lthtmlinlinemathZ
\lthtmlcheckvsize\clearpage}

{\newpage\clearpage
\lthtmlinlinemathA{tex2html_wrap_inline5678}%
$ \underline v_+$%
\lthtmlinlinemathZ
\lthtmlcheckvsize\clearpage}

{\newpage\clearpage
\lthtmlinlinemathA{tex2html_wrap_inline5680}%
$ L^2(e)$%
\lthtmlinlinemathZ
\lthtmlcheckvsize\clearpage}

{\newpage\clearpage
\lthtmlinlinemathA{tex2html_wrap_inline5684}%
$ H^1$%
\lthtmlinlinemathZ
\lthtmlcheckvsize\clearpage}

\stepcounter{subsection}
{\newpage\clearpage
\lthtmlinlinemathA{tex2html_wrap_inline5694}%
$ \underline u \in H_0(\mathop{\rm curl};\Omega )$%
\lthtmlinlinemathZ
\lthtmlcheckvsize\clearpage}

{\newpage\clearpage
\lthtmlinlinemathA{tex2html_wrap_inline5696}%
$ \underline v \in H_0(\mathop{\rm curl};\Omega )$%
\lthtmlinlinemathZ
\lthtmlcheckvsize\clearpage}

{\newpage\clearpage
\lthtmlinlinemathA{tex2html_wrap_indisplay5698}%
$\displaystyle \int_{\Omega } \mathop{\rm curl}\underline u\, \mathop{\rm curl}\underline v\,dx + \int_{\Omega } c(x)\, \underline u\, \cdot \underline v\,dx =  \int_{\Omega } \underline f\, \cdot \underline v\,dx$%
\lthtmlindisplaymathZ
\lthtmlcheckvsize\clearpage}

{\newpage\clearpage
\lthtmldisplayA{displaymath5701}%
\begin{displaymath}\begin{split} 	a(\underline u,\underline v) &:= \int_{\Omega } \mathop{\rm curl}\underline u\, \mathop{\rm curl}\underline v\,dx + \int_{\Omega } c(x)\, \underline u\, \cdot \underline v\,dx \\  	l(\underline v) & := \int_{\Omega } \underline f\, \cdot \underline v\,dx \end{split}\end{displaymath}%
\lthtmldisplayZ
\lthtmlcheckvsize\clearpage}

{\newpage\clearpage
\lthtmlinlinemathA{tex2html_wrap_inline5703}%
$ a(\cdot,\cdot)$%
\lthtmlinlinemathZ
\lthtmlcheckvsize\clearpage}

{\newpage\clearpage
\lthtmlinlinemathA{tex2html_wrap_inline5705}%
$ H_0(\mathop{\rm curl};\Omega )\times H_0(\mathop{\rm curl};\Omega )$%
\lthtmlinlinemathZ
\lthtmlcheckvsize\clearpage}

\stepcounter{section}
{\newpage\clearpage
\lthtmlinlinemathA{tex2html_wrap_inline5717}%
$ \hat{K}$%
\lthtmlinlinemathZ
\lthtmlcheckvsize\clearpage}

{\newpage\clearpage
\lthtmlinlinemathA{tex2html_wrap_inline5719}%
$ F_K(\hat{x})$%
\lthtmlinlinemathZ
\lthtmlcheckvsize\clearpage}

{\newpage\clearpage
\lthtmlinlinemathA{tex2html_wrap_inline5721}%
$ K = F_K(\hat{K})$%
\lthtmlinlinemathZ
\lthtmlcheckvsize\clearpage}

{\newpage\clearpage
\lthtmlinlinemathA{tex2html_wrap_inline5723}%
$ \hat{R}$%
\lthtmlinlinemathZ
\lthtmlcheckvsize\clearpage}

{\newpage\clearpage
\lthtmlinlinemathA{tex2html_wrap_inline5727}%
$ R_K$%
\lthtmlinlinemathZ
\lthtmlcheckvsize\clearpage}

{\newpage\clearpage
\lthtmlinlinemathA{tex2html_wrap_inline5729}%
$ K$%
\lthtmlinlinemathZ
\lthtmlcheckvsize\clearpage}

{\newpage\clearpage
\lthtmlinlinemathA{tex2html_wrap_inline5731}%
$ \mathcal{A} = \{\alpha_i(\cdot)\}_{i=1}^N$%
\lthtmlinlinemathZ
\lthtmlcheckvsize\clearpage}

{\newpage\clearpage
\lthtmlinlinemathA{tex2html_wrap_inline5735}%
$ N < \infty$%
\lthtmlinlinemathZ
\lthtmlcheckvsize\clearpage}

{\newpage\clearpage
\lthtmlinlinemathA{tex2html_wrap_inline5739}%
$ \mathcal{A}$%
\lthtmlinlinemathZ
\lthtmlcheckvsize\clearpage}

{\newpage\clearpage
\lthtmlinlinemathA{tex2html_wrap_inline5741}%
$ \alpha_i(\cdot)$%
\lthtmlinlinemathZ
\lthtmlcheckvsize\clearpage}

\stepcounter{subsection}
\stepcounter{subsubsection}
{\newpage\clearpage
\lthtmlinlinemathA{tex2html_wrap_inline5749}%
$ \hat{R} = \mathcal{R}^k$%
\lthtmlinlinemathZ
\lthtmlcheckvsize\clearpage}

{\newpage\clearpage
\lthtmlinlinemathA{tex2html_wrap_inline5751}%
$ \mathbb{P}_k(\hat{\Sigma})$%
\lthtmlinlinemathZ
\lthtmlcheckvsize\clearpage}

{\newpage\clearpage
\lthtmlinlinemathA{tex2html_wrap_inline5753}%
$ k$%
\lthtmlinlinemathZ
\lthtmlcheckvsize\clearpage}

{\newpage\clearpage
\lthtmlinlinemathA{tex2html_wrap_inline5755}%
$ \hat{\Sigma}$%
\lthtmlinlinemathZ
\lthtmlcheckvsize\clearpage}

{\newpage\clearpage
\lthtmlinlinemathA{tex2html_wrap_inline5759}%
$ \tilde{\mathbb{P}}_k$%
\lthtmlinlinemathZ
\lthtmlcheckvsize\clearpage}

{\newpage\clearpage
\lthtmlinlinemathA{tex2html_wrap_inline5765}%
$ d$%
\lthtmlinlinemathZ
\lthtmlcheckvsize\clearpage}

{\newpage\clearpage
\lthtmlinlinemathA{tex2html_wrap_indisplay5770}%
$\displaystyle \mathcal{S}^k := \{\, \underline p \in (\tilde{\mathbb{P}}_k)^d : \underline p \cdot \hat{\underline x} = \sum_{i=1}^{d} p_i\,\hat{x}_i \equiv 0 \,\}\,,$%
\lthtmlindisplaymathZ
\lthtmlcheckvsize\clearpage}

{\newpage\clearpage
\lthtmlinlinemathA{tex2html_wrap_inline5772}%
$ \hat{x} \in \hat{K}$%
\lthtmlinlinemathZ
\lthtmlcheckvsize\clearpage}

{\newpage\clearpage
\lthtmlinlinemathA{tex2html_wrap_inline5778}%
$ k(k+2)$%
\lthtmlinlinemathZ
\lthtmlcheckvsize\clearpage}

{\newpage\clearpage
\lthtmlinlinemathA{tex2html_wrap_indisplay5785}%
$\displaystyle \mathcal{R}^k = \left(\mathbb{P}_{k-1}(\hat{K}) \right)^d \oplus \mathcal{S}^k\,.$%
\lthtmlindisplaymathZ
\lthtmlcheckvsize\clearpage}

{\newpage\clearpage
\lthtmldisplayA{displaymath5787}%
\begin{displaymath}\begin{split} 					\mathrm{dim} (\mathcal{R}^k) &= k(k+2) \qquad \textrm{for} \quad d=2\,, \\  					\mathrm{dim} (\mathcal{R}^k) &= \frac{(k+3)(k+2)k}{2} \qquad \textrm{for} \quad d=3\,. 				\end{split}\end{displaymath}%
\lthtmldisplayZ
\lthtmlcheckvsize\clearpage}

{\newpage\clearpage
\lthtmlinlinemathA{tex2html_wrap_inline5789}%
$ \mathcal{R}^k$%
\lthtmlinlinemathZ
\lthtmlcheckvsize\clearpage}

{\newpage\clearpage
\lthtmlinlinemathA{tex2html_wrap_indisplay5791}%
$\displaystyle \mathcal{R}^k = \left(\mathbb{P}_{k-1}(\hat{K}) \right)^2 \oplus \tilde{\mathbb{P}}_{k-1}\, \left(\begin{array}{cc} \hat{x}_2 \\  -\hat{x}_1 				\end{array}\right)\,.$%
\lthtmlindisplaymathZ
\lthtmlcheckvsize\clearpage}

{\newpage\clearpage
\lthtmlinlinemathA{tex2html_wrap_indisplay5795}%
$\displaystyle \tilde{\mathbb{P}}_{k-1}\, \left(\begin{array}{cc} \hat{x}_2 \\  -\hat{x}_1 \end{array}\right) \subseteq \mathcal{S}^k
%%\left\{\v p\quad \big|\quad \v p = \tilde{p} \left(\begin{array}{cc} x_2 \\-x_1 \end{array}\right)\,,\, \tilde{p}
				$%
\lthtmlindisplaymathZ
\lthtmlcheckvsize\clearpage}

{\newpage\clearpage
\lthtmlinlinemathA{tex2html_wrap_inline5797}%
$ \tilde{\mathbb{P}}_{k-1}$%
\lthtmlinlinemathZ
\lthtmlcheckvsize\clearpage}

{\newpage\clearpage
\lthtmlinlinemathA{tex2html_wrap_inline5799}%
$ k-1$%
\lthtmlinlinemathZ
\lthtmlcheckvsize\clearpage}

{\newpage\clearpage
\lthtmlinlinemathA{tex2html_wrap_inline5803}%
$ \mathcal{S}^k$%
\lthtmlinlinemathZ
\lthtmlcheckvsize\clearpage}

{\newpage\clearpage
\lthtmlinlinemathA{tex2html_wrap_inline5809}%
$ k=1$%
\lthtmlinlinemathZ
\lthtmlcheckvsize\clearpage}

{\newpage\clearpage
\lthtmlinlinemathA{tex2html_wrap_inline5811}%
$ k=2$%
\lthtmlinlinemathZ
\lthtmlcheckvsize\clearpage}

{\newpage\clearpage
\lthtmlinlinemathA{tex2html_wrap_indisplay5814}%
$\displaystyle \mathcal{R}^1 = \left\langle   								\left(\begin{array}{cc} 1 \\  0 \end{array}\right)\,_, 								\left(\begin{array}{cc} 0 \\  1 \end{array}\right)\,_, 								\left(\begin{array}{cc} \hat{x}_2 \\  -\hat{x}_1 \end{array}\right) \right\rangle$%
\lthtmlindisplaymathZ
\lthtmlcheckvsize\clearpage}

{\newpage\clearpage
\lthtmlinlinemathA{tex2html_wrap_indisplay5816}%
$\displaystyle \mathcal{R}^2 = \left(\mathbb{P}_{1}(\hat{K}) \right)^2 \oplus 
\left\langle 
\left(\begin{array}{cc} \hat{x}_1\,\hat{x}_2 \\  -{\hat{x}_1}^2 \end{array}\right)\,_,
\left(\begin{array}{cc} {\hat{x}_2}^2 \\  -\hat{x}_1\,\hat{x}_2 \end{array}\right)
\right\rangle					
$%
\lthtmlindisplaymathZ
\lthtmlcheckvsize\clearpage}

{\newpage\clearpage
\lthtmlinlinemathA{tex2html_wrap_inline5823}%
$ \mathcal{S}^1$%
\lthtmlinlinemathZ
\lthtmlcheckvsize\clearpage}

{\newpage\clearpage
\lthtmlinlinemathA{tex2html_wrap_inline5825}%
$ \underline p$%
\lthtmlinlinemathZ
\lthtmlcheckvsize\clearpage}

{\newpage\clearpage
\lthtmlinlinemathA{tex2html_wrap_inline5827}%
$ (\mathbb{P}_{1}(\hat{K}))^3$%
\lthtmlinlinemathZ
\lthtmlcheckvsize\clearpage}

{\newpage\clearpage
\lthtmlinlinemathA{tex2html_wrap_indisplay5829}%
$\displaystyle p_i = \sum_{j=1}^3 a_{ij} \hat{x}_j\,, \qquad i=1,2,3\,.
$%
\lthtmlindisplaymathZ
\lthtmlcheckvsize\clearpage}

{\newpage\clearpage
\lthtmlinlinemathA{tex2html_wrap_indisplay5835}%
$\displaystyle \underline p \cdot \hat{\underline x} = 
\sum_{i=1}^3 a_{ii}\hat{x}_i^2 + \sum_{\substack{i,j=1 \\  j>i}}^3 (a_{ij}+a_{ji})\hat{x}_i \hat{x}_j \equiv 0\,.
$%
\lthtmlindisplaymathZ
\lthtmlcheckvsize\clearpage}

{\newpage\clearpage
\lthtmldisplayA{displaymath5839}%
\begin{displaymath}\begin{split} 					&a_{11}=a_{22}=a_{33} = 0 \\  					&a_{12}= - a_{21}\,,\quad a_{13}= - a_{31}\,,\quad a_{23}= - a_{32}\,. 				\end{split}\end{displaymath}%
\lthtmldisplayZ
\lthtmlcheckvsize\clearpage}

{\newpage\clearpage
\lthtmlinlinemathA{tex2html_wrap_inline5843}%
$ a_{ij} = 1$%
\lthtmlinlinemathZ
\lthtmlcheckvsize\clearpage}

{\newpage\clearpage
\lthtmlinlinemathA{tex2html_wrap_inline5845}%
$ i=1,2,3$%
\lthtmlinlinemathZ
\lthtmlcheckvsize\clearpage}

{\newpage\clearpage
\lthtmlinlinemathA{tex2html_wrap_inline5847}%
$ j>i$%
\lthtmlinlinemathZ
\lthtmlcheckvsize\clearpage}

{\newpage\clearpage
\lthtmlinlinemathA{tex2html_wrap_indisplay5849}%
$\displaystyle \mathcal{R}^1 = \left(\mathbb{P}_{0}(\hat{K}) \right)^3 \oplus  													 \left\langle  													 	\left(\begin{array}{ccc} 0 \\  \hat{x}_3 \\  \hat{x}_2 \end{array}\right)\,_,\, 														\left(\begin{array}{ccc} \hat{x}_3 \\  0 \\  \hat{x}_1 \end{array}\right)\,_,\, 														\left(\begin{array}{ccc} \hat{x}_2 \\  \hat{x}_1 \\  0 \end{array}\right) 														\right\rangle$%
\lthtmlindisplaymathZ
\lthtmlcheckvsize\clearpage}

{\newpage\clearpage
\lthtmlinlinemathA{tex2html_wrap_inline5853}%
$ (\mathbb{P}_{k}(\hat{K}) )^d$%
\lthtmlinlinemathZ
\lthtmlcheckvsize\clearpage}

\stepcounter{subsubsection}
{\newpage\clearpage
\lthtmlinlinemathA{tex2html_wrap_inline5868}%
$ n$%
\lthtmlinlinemathZ
\lthtmlcheckvsize\clearpage}

{\newpage\clearpage
\lthtmlinlinemathA{tex2html_wrap_inline5870}%
$ n+k+2 \choose n$%
\lthtmlinlinemathZ
\lthtmlcheckvsize\clearpage}

{\newpage\clearpage
\lthtmlinlinemathA{tex2html_wrap_inline5875}%
$ \hat{\underline t}$%
\lthtmlinlinemathZ
\lthtmlcheckvsize\clearpage}

{\newpage\clearpage
\lthtmlinlinemathA{tex2html_wrap_indisplay5883}%
$\displaystyle \hat{\alpha}(\hat{\underline u}) := \int_{\hat{e}} (\hat{\underline t}\cdot \hat{\underline u})\, \hat{\varphi }\,d\hat{s} \quad \forall \hat{\varphi }  						\in \mathbb{P}_{k-1}(\hat{e})\,,$%
\lthtmlindisplaymathZ
\lthtmlcheckvsize\clearpage}

{\newpage\clearpage
\lthtmlinlinemathA{tex2html_wrap_inline5885}%
$ \hat{e}$%
\lthtmlinlinemathZ
\lthtmlcheckvsize\clearpage}

{\newpage\clearpage
\lthtmlinlinemathA{tex2html_wrap_inline5889}%
$ 3k$%
\lthtmlinlinemathZ
\lthtmlcheckvsize\clearpage}

{\newpage\clearpage
\lthtmlinlinemathA{tex2html_wrap_indisplay5891}%
$\displaystyle \hat{\alpha}(\hat{\underline u}) := \int_{\hat{K}} \hat{\underline u}\cdot \hat{\underline \varphi }\,d\hat{x} \quad \forall \hat{\underline \varphi }  						\in (\mathbb{P}_{k-2}(\hat{K}))^2\,.$%
\lthtmlindisplaymathZ
\lthtmlcheckvsize\clearpage}

{\newpage\clearpage
\lthtmlinlinemathA{tex2html_wrap_inline5893}%
$ k(k-1)$%
\lthtmlinlinemathZ
\lthtmlcheckvsize\clearpage}

{\newpage\clearpage
\lthtmlinlinemathA{tex2html_wrap_inline5900}%
$ \hat{\underline n}$%
\lthtmlinlinemathZ
\lthtmlcheckvsize\clearpage}

{\newpage\clearpage
\lthtmlinlinemathA{tex2html_wrap_inline5915}%
$ 6k$%
\lthtmlinlinemathZ
\lthtmlcheckvsize\clearpage}

{\newpage\clearpage
\lthtmlinlinemathA{tex2html_wrap_indisplay5917}%
$\displaystyle \hat{\alpha}(\hat{\underline u}) := \int_{\hat{f}} (\hat{\underline u}\wedge \hat{\underline n})\cdot \hat{\underline \varphi }\,d\hat{a} \quad  						\forall \hat{\varphi } \in (\mathbb{P}_{k-2}(\hat{f}) )^2\,,$%
\lthtmlindisplaymathZ
\lthtmlcheckvsize\clearpage}

{\newpage\clearpage
\lthtmlinlinemathA{tex2html_wrap_inline5919}%
$ \hat{f}$%
\lthtmlinlinemathZ
\lthtmlcheckvsize\clearpage}

{\newpage\clearpage
\lthtmlinlinemathA{tex2html_wrap_inline5923}%
$ 4k(k-1)$%
\lthtmlinlinemathZ
\lthtmlcheckvsize\clearpage}

{\newpage\clearpage
\lthtmlinlinemathA{tex2html_wrap_indisplay5925}%
$\displaystyle \hat{\alpha}(\hat{\underline u}) := \int_{\hat{K}} \hat{\underline u}\cdot \hat{\underline \varphi }\,d\hat{x} \quad  						\forall \hat{\underline \varphi } \in (\mathbb{P}_{k-3}(\hat{K}))^3\,.$%
\lthtmlindisplaymathZ
\lthtmlcheckvsize\clearpage}

{\newpage\clearpage
\lthtmlinlinemathA{tex2html_wrap_inline5927}%
$ \frac{k(k-1)(k-2)}{2}$%
\lthtmlinlinemathZ
\lthtmlcheckvsize\clearpage}

{\newpage\clearpage
\lthtmlinlinemathA{tex2html_wrap_inline5933}%
$ k\leq3$%
\lthtmlinlinemathZ
\lthtmlcheckvsize\clearpage}

{\newpage\clearpage
\lthtmlinlinemathA{tex2html_wrap_inline5944}%
$ \hat{\underline u}\in \mathcal{R}^k$%
\lthtmlinlinemathZ
\lthtmlcheckvsize\clearpage}

{\newpage\clearpage
\lthtmlinlinemathA{tex2html_wrap_inline5946}%
$ \hat{\alpha}(\hat{\underline u})$%
\lthtmlinlinemathZ
\lthtmlcheckvsize\clearpage}

{\newpage\clearpage
\lthtmlinlinemathA{tex2html_wrap_inline5949}%
$ \hat{K} = \left\{ (\hat{x},\hat{y})\in\mathbb{R}^2:\quad 0\leq \hat{x}\leq 1\,,\,\, 0\leq \hat{y}\leq 1-\hat{x} \right\}$%
\lthtmlinlinemathZ
\lthtmlcheckvsize\clearpage}

{\newpage\clearpage
\lthtmlinlinemathA{tex2html_wrap_inline5951}%
$ \hat{e}_0 = \overline{(0,0),(1,0)}$%
\lthtmlinlinemathZ
\lthtmlcheckvsize\clearpage}

{\newpage\clearpage
\lthtmlinlinemathA{tex2html_wrap_indisplay5953}%
$\displaystyle \hat{\underline t}_0 = \left(\begin{array}{cc} 1 \\  0 \end{array}\right)\,,\quad
\hat{\underline t}_1 = \frac{1}{\sqrt{2}}\left(\begin{array}{cc} -1 \\  1 \end{array}\right)\,,\quad
\hat{\underline t}_2 = \left(\begin{array}{cc} 0 \\  -1 \end{array}\right)\,.
$%
\lthtmlindisplaymathZ
\lthtmlcheckvsize\clearpage}

{\newpage\clearpage
\lthtmlinlinemathA{tex2html_wrap_inline5955}%
$ \mathcal{R}^1$%
\lthtmlinlinemathZ
\lthtmlcheckvsize\clearpage}

{\newpage\clearpage
\lthtmlinlinemathA{tex2html_wrap_inline5961}%
$ \int_{\hat{e}_i} (\hat{\underline t}\cdot \hat{\underline u})\, \hat{\varphi }\,d\hat{s}\,, \forall \hat{\varphi } \in \mathbb{P}_{0}(\hat{e}_i)$%
\lthtmlinlinemathZ
\lthtmlcheckvsize\clearpage}

{\newpage\clearpage
\lthtmlinlinemathA{tex2html_wrap_inline5963}%
$ \varphi \equiv 1$%
\lthtmlinlinemathZ
\lthtmlcheckvsize\clearpage}

{\newpage\clearpage
\lthtmlinlinemathA{tex2html_wrap_inline5965}%
$ \mathbb{P}_{0}(\hat{e}_i)$%
\lthtmlinlinemathZ
\lthtmlcheckvsize\clearpage}

{\newpage\clearpage
\lthtmlinlinemathA{tex2html_wrap_indisplay5967}%
$\displaystyle \hat{\alpha}_i(\hat{\underline u}) = \int_{\hat{e}_i} (\hat{\underline t}\cdot \hat{\underline u})\,d\hat{s} \quad i=0,1,2\,.$%
\lthtmlindisplaymathZ
\lthtmlcheckvsize\clearpage}

{\newpage\clearpage
\lthtmlinlinemathA{tex2html_wrap_inline5969}%
$ \hat{\underline N}_0,\hat{\underline N}_1,\hat{\underline N}_2$%
\lthtmlinlinemathZ
\lthtmlcheckvsize\clearpage}

{\newpage\clearpage
\lthtmlinlinemathA{tex2html_wrap_inline5973}%
$ \hat{\alpha}_i(\hat{\underline N}_j) = \delta_{ij}$%
\lthtmlinlinemathZ
\lthtmlcheckvsize\clearpage}

{\newpage\clearpage
\lthtmlinlinemathA{tex2html_wrap_inline5975}%
$ \hat{\underline N}_i$%
\lthtmlinlinemathZ
\lthtmlcheckvsize\clearpage}

{\newpage\clearpage
\lthtmlinlinemathA{tex2html_wrap_indisplay5979}%
$\displaystyle \hat{\underline N}_0 = \left(\begin{array}{cc} 1-\hat{y} \\  \hat{x} \end{array}\right)\,,\quad 					\hat{\underline N}_1 = \left(\begin{array}{cc} -\hat{y} \\  \hat{x} \end{array}\right)\,,\quad 					\hat{\underline N}_2 = \left(\begin{array}{cc} -\hat{y} \\  \hat{x}-1 \end{array}\right)\,.$%
\lthtmlindisplaymathZ
\lthtmlcheckvsize\clearpage}

\stepcounter{subsubsection}
{\newpage\clearpage
\lthtmlinlinemathA{tex2html_wrap_indisplay5984}%
$\displaystyle K \ni x = F_K(\hat{x}) = B_K \hat{x} + b_K
$%
\lthtmlindisplaymathZ
\lthtmlcheckvsize\clearpage}

{\newpage\clearpage
\lthtmlinlinemathA{tex2html_wrap_inline5988}%
$ N_i$%
\lthtmlinlinemathZ
\lthtmlcheckvsize\clearpage}

{\newpage\clearpage
\lthtmlinlinemathA{tex2html_wrap_inline5992}%
$ \hat{N}_i$%
\lthtmlinlinemathZ
\lthtmlcheckvsize\clearpage}

{\newpage\clearpage
\lthtmlinlinemathA{tex2html_wrap_indisplay5996}%
$\displaystyle N_i(x) = \left( \hat{N}_i \circ F_K^{-1} \right)(x)$%
\lthtmlindisplaymathZ
\lthtmlcheckvsize\clearpage}

{\newpage\clearpage
\lthtmlinlinemathA{tex2html_wrap_inline6000}%
$ H(\mathop{\rm curl};\hat{K})$%
\lthtmlinlinemathZ
\lthtmlcheckvsize\clearpage}

{\newpage\clearpage
\lthtmlinlinemathA{tex2html_wrap_inline6002}%
$ H(\mathop{\rm curl}; K)$%
\lthtmlinlinemathZ
\lthtmlcheckvsize\clearpage}

{\newpage\clearpage
\lthtmlinlinemathA{tex2html_wrap_inline6011}%
$ \underline N_i(x)$%
\lthtmlinlinemathZ
\lthtmlcheckvsize\clearpage}

{\newpage\clearpage
\lthtmlinlinemathA{tex2html_wrap_indisplay6015}%
$\displaystyle \underline N_i(x) = \mathcal{P}_K (\hat{\underline N}_i) =  \left(\hat{D}F_K^{-T} \hat{\underline N}_i\right) \circ F_K^{-1} (x)\,,$%
\lthtmlindisplaymathZ
\lthtmlcheckvsize\clearpage}

{\newpage\clearpage
\lthtmlinlinemathA{tex2html_wrap_inline6017}%
$ \hat{D}F_K$%
\lthtmlinlinemathZ
\lthtmlcheckvsize\clearpage}

{\newpage\clearpage
\lthtmlinlinemathA{tex2html_wrap_inline6019}%
$ \frac{d}{d\hat{x}}F_K(\hat{x})$%
\lthtmlinlinemathZ
\lthtmlcheckvsize\clearpage}

{\newpage\clearpage
\lthtmlinlinemathA{tex2html_wrap_inline6022}%
$ H(\mathop{\rm div}; \Omega )$%
\lthtmlinlinemathZ
\lthtmlcheckvsize\clearpage}

{\newpage\clearpage
\lthtmlinlinemathA{tex2html_wrap_inline6026}%
$ F_K(\hat{x}) = B_K \hat{x} + b_k$%
\lthtmlinlinemathZ
\lthtmlcheckvsize\clearpage}

{\newpage\clearpage
\lthtmlinlinemathA{tex2html_wrap_inline6030}%
$ B_K$%
\lthtmlinlinemathZ
\lthtmlcheckvsize\clearpage}

{\newpage\clearpage
\lthtmlinlinemathA{tex2html_wrap_indisplay6032}%
$\displaystyle \underline v(x) = \mathcal{P}_K (\hat{\underline v}) =  B_K^{-T} \left(\hat{\underline v} \circ F_K^{-1} \right)(x)\,,$%
\lthtmlindisplaymathZ
\lthtmlcheckvsize\clearpage}

\stepcounter{subsubsection}
{\newpage\clearpage
\lthtmlinlinemathA{tex2html_wrap_inline6035}%
$ \Omega \subset\mathbb{R}^2$%
\lthtmlinlinemathZ
\lthtmlcheckvsize\clearpage}

{\newpage\clearpage
\lthtmlinlinemathA{tex2html_wrap_indisplay6041}%
$\displaystyle B_K^{-T} = \det B_K^{-1}\,R^T B_K\,R \,,$%
\lthtmlindisplaymathZ
\lthtmlcheckvsize\clearpage}

{\newpage\clearpage
\lthtmlinlinemathA{tex2html_wrap_inline6043}%
$ R$%
\lthtmlinlinemathZ
\lthtmlcheckvsize\clearpage}

{\newpage\clearpage
\lthtmlinlinemathA{tex2html_wrap_inline6048}%
$ \underline v(x) = \mathcal{P}_K(\hat{\underline v})$%
\lthtmlinlinemathZ
\lthtmlcheckvsize\clearpage}

{\newpage\clearpage
\lthtmlinlinemathA{tex2html_wrap_inline6050}%
$ \varphi (x) = \left( \hat{\varphi }\circ F_K^{-1} \right)(x)$%
\lthtmlinlinemathZ
\lthtmlcheckvsize\clearpage}

{\newpage\clearpage
\lthtmlinlinemathA{tex2html_wrap_inline6052}%
$ \hat{x} = 
F_K^{-1}(x)$%
\lthtmlinlinemathZ
\lthtmlcheckvsize\clearpage}

{\newpage\clearpage
\lthtmlinlinemathA{tex2html_wrap_inline6054}%
$ F_K$%
\lthtmlinlinemathZ
\lthtmlcheckvsize\clearpage}

{\newpage\clearpage
\lthtmlinlinemathA{tex2html_wrap_inline6058}%
$ D\underline v$%
\lthtmlinlinemathZ
\lthtmlcheckvsize\clearpage}

{\newpage\clearpage
\lthtmlinlinemathA{tex2html_wrap_indisplay6060}%
$\displaystyle D\underline v = B_K^{-T}\, \hat{D}\hat{\underline v}\, B_K^{-1}\, .$%
\lthtmlindisplaymathZ
\lthtmlcheckvsize\clearpage}

{\newpage\clearpage
\lthtmlinlinemathA{tex2html_wrap_indisplay6062}%
$\displaystyle \mathop{\rm curl}\underline v = \det B_K^{-1} \widehat{\mathop{\rm curl}} \hat{\underline v}\, .$%
\lthtmlindisplaymathZ
\lthtmlcheckvsize\clearpage}

{\newpage\clearpage
\lthtmlinlinemathA{tex2html_wrap_inline6074}%
$ R\,Dv$%
\lthtmlinlinemathZ
\lthtmlcheckvsize\clearpage}

{\newpage\clearpage
\lthtmlinlinemathA{tex2html_wrap_inline6076}%
$ B_K^{-T}$%
\lthtmlinlinemathZ
\lthtmlcheckvsize\clearpage}

{\newpage\clearpage
\lthtmlinlinemathA{tex2html_wrap_inline6080}%
$ \det B_K^{-1}\,R\,\hat{D}\hat{v}$%
\lthtmlinlinemathZ
\lthtmlcheckvsize\clearpage}

{\newpage\clearpage
\lthtmlinlinemathA{tex2html_wrap_indisplay6085}%
$\displaystyle \int_{K} \mathop{\rm curl}\underline v\, \varphi \, dx = \int_{\hat{K}} \widehat{\mathop{\rm curl}}\hat{\underline v}\, \hat{\varphi } \,d\hat{x}\,,$%
\lthtmlindisplaymathZ
\lthtmlcheckvsize\clearpage}

{\newpage\clearpage
\lthtmlinlinemathA{tex2html_wrap_indisplay6087}%
$\displaystyle \int_{K} \mathop{\rm curl}\underline v\, \mathop{\rm curl}\underline u \, dx = | B_K |^{-1}\,\int_{\hat{K}} \widehat{\mathop{\rm curl}}\hat{\underline v}\, 						\widehat{\mathop{\rm curl}}\hat{\underline u}  \,d\hat{x}\,.$%
\lthtmlindisplaymathZ
\lthtmlcheckvsize\clearpage}

\stepcounter{subsubsection}
{\newpage\clearpage
\lthtmlinlinemathA{tex2html_wrap_inline6092}%
$ \hat{\underline v}$%
\lthtmlinlinemathZ
\lthtmlcheckvsize\clearpage}

{\newpage\clearpage
\lthtmlinlinemathA{tex2html_wrap_inline6096}%
$ \mathop{\rm Curl}v$%
\lthtmlinlinemathZ
\lthtmlcheckvsize\clearpage}

{\newpage\clearpage
\lthtmlinlinemathA{tex2html_wrap_indisplay6098}%
$\displaystyle \left({\mathop{\rm Curl}v}\right)_{ij} = \frac{\partial v_j}{\partial x_i} - \frac{\partial v_i}{\partial x_j}$%
\lthtmlindisplaymathZ
\lthtmlcheckvsize\clearpage}

{\newpage\clearpage
\lthtmlinlinemathA{tex2html_wrap_inline6100}%
$ \mathop{\rm Curl}v = D\underline v^T - D\underline v$%
\lthtmlinlinemathZ
\lthtmlcheckvsize\clearpage}

{\newpage\clearpage
\lthtmlinlinemathA{tex2html_wrap_indisplay6102}%
$\displaystyle \mathop{\rm Curl}v = B_K^{-T} \widehat{\mathop{\rm Curl}}\,\hat{v} \,B_K^{-1}$%
\lthtmlindisplaymathZ
\lthtmlcheckvsize\clearpage}

{\newpage\clearpage
\lthtmlinlinemathA{tex2html_wrap_inline6109}%
$ \underline v(x)$%
\lthtmlinlinemathZ
\lthtmlcheckvsize\clearpage}

{\newpage\clearpage
\lthtmlinlinemathA{tex2html_wrap_inline6113}%
$ \hat{\underline v}(\hat{x})$%
\lthtmlinlinemathZ
\lthtmlcheckvsize\clearpage}

{\newpage\clearpage
\lthtmlinlinemathA{tex2html_wrap_indisplay6115}%
$\displaystyle \left(\mathop{\rm curl}\underline v\right)_i(x) = \det \mathrm{M_i}(x) \,, \qquad i= 1,2,3\,$%
\lthtmlindisplaymathZ
\lthtmlcheckvsize\clearpage}

{\newpage\clearpage
\lthtmlinlinemathA{tex2html_wrap_inline6117}%
$ \mathrm{M_i}$%
\lthtmlinlinemathZ
\lthtmlcheckvsize\clearpage}

{\newpage\clearpage
\lthtmlinlinemathA{tex2html_wrap_inline6119}%
$ D(F_K^{-1}) = B_K^{-1}$%
\lthtmlinlinemathZ
\lthtmlcheckvsize\clearpage}

{\newpage\clearpage
\lthtmlinlinemathA{tex2html_wrap_inline6121}%
$ (\widehat{\mathop{\rm curl}}\,\hat{\underline v}\circ F_K^{-1})(x)$%
\lthtmlinlinemathZ
\lthtmlcheckvsize\clearpage}

{\newpage\clearpage
\lthtmlinlinemathA{tex2html_wrap_indisplay6123}%
$\displaystyle \left(\mathrm{M_i}\right)_{kl}(x) := \left\{ \begin{array}{ll}  																(\widehat{\mathop{\rm curl}}\,\hat{v} \circ F_K^{-1})_k (x) & \textrm{if} \quad l=i \\  																(B_K^{-1})_{kl} & \textrm{if} \quad l\neq i 												\end{array} \right.$%
\lthtmlindisplaymathZ
\lthtmlcheckvsize\clearpage}

{\newpage\clearpage
\lthtmlinlinemathA{tex2html_wrap_indisplay6125}%
$\displaystyle \mathop{\rm curl}\underline v = \left( \begin{array}{ccc} ({\mathop{\rm Curl}v})_{23} \\  ({\mathop{\rm Curl}v})_{31}  \\  ({\mathop{\rm Curl}v})_{12} \end{array}\right)\,.$%
\lthtmlindisplaymathZ
\lthtmlcheckvsize\clearpage}

{\newpage\clearpage
\lthtmlinlinemathA{tex2html_wrap_inline6127}%
$ (\mathop{\rm curl}\underline v)_1 = {\mathop{\rm Curl}v}_{23}$%
\lthtmlinlinemathZ
\lthtmlcheckvsize\clearpage}

{\newpage\clearpage
\lthtmlinlinemathA{tex2html_wrap_inline6129}%
$ b_{ij} := (B_K^{-1})_{ij}$%
\lthtmlinlinemathZ
\lthtmlcheckvsize\clearpage}

{\newpage\clearpage
\lthtmlinlinemathA{tex2html_wrap_indisplay6131}%
$\displaystyle ({\mathop{\rm Curl}v})_{23} = b_{k2}\, (\widehat{\mathop{\rm Curl}}\, \hat{v})_{kl}\,b_{l3}
$%
\lthtmlindisplaymathZ
\lthtmlcheckvsize\clearpage}

{\newpage\clearpage
\lthtmlinlinemathA{tex2html_wrap_indisplay6135}%
$\displaystyle ({\mathop{\rm Curl}v})_{23} = (b_{12}b_{23} - b_{22}b_{13})(\widehat{\mathop{\rm Curl}} \,\hat{v})_{12}
-(b_{12}b_{33} - b_{32}b_{13})(\widehat{\mathop{\rm Curl}}\, \hat{v})_{31}
+(b_{22}b_{33} - b_{32}b_{23})(\widehat{\mathop{\rm Curl}}\, \hat{v})_{23}\,,
$%
\lthtmlindisplaymathZ
\lthtmlcheckvsize\clearpage}

{\newpage\clearpage
\lthtmlinlinemathA{tex2html_wrap_indisplay6137}%
$\displaystyle \mathrm{M_1} := \left(\begin{array}{ccc} (\widehat{\mathop{\rm curl}}\, v)_1 & b_{12} & b_{13} \\ 
(\widehat{\mathop{\rm curl}}\, v)_2 & b_{22} & b_{23} \\ 	
(\widehat{\mathop{\rm curl}}\, v)_3 & b_{32} & b_{33} 
\end{array}\right) \,.
$%
\lthtmlindisplaymathZ
\lthtmlcheckvsize\clearpage}

{\newpage\clearpage
\lthtmlinlinemathA{tex2html_wrap_indisplay6148}%
$\displaystyle \mathop{\rm curl}\underline v = \frac{1}{\det B_K}\,B_K\,(\widehat{\mathop{\rm curl}}\,\hat{\underline v} \circ F_K^{-1})\,.$%
\lthtmlindisplaymathZ
\lthtmlcheckvsize\clearpage}

{\newpage\clearpage
\lthtmlinlinemathA{tex2html_wrap_inline6150}%
$ (\mathop{\rm curl}\underline v)_1$%
\lthtmlinlinemathZ
\lthtmlcheckvsize\clearpage}

{\newpage\clearpage
\lthtmlinlinemathA{tex2html_wrap_indisplay6152}%
$\displaystyle (\mathop{\rm curl}\underline v)_1 = \frac{1}{\det B_K} (B_K)_{1j} ((\widehat{\mathop{\rm curl}}\,\hat{\underline v})_j \circ F^{-1})\,.$%
\lthtmlindisplaymathZ
\lthtmlcheckvsize\clearpage}

{\newpage\clearpage
\lthtmlinlinemathA{tex2html_wrap_inline6154}%
$ \det \mathrm{M_1}$%
\lthtmlinlinemathZ
\lthtmlcheckvsize\clearpage}

{\newpage\clearpage
\lthtmlinlinemathA{tex2html_wrap_indisplay6158}%
$\displaystyle \det  \mathrm{M_1} = (\widehat{\mathop{\rm curl}}\, \hat{\underline v})_1 \det\mathcal{B}^{inv}_{11} 
-(\widehat{\mathop{\rm curl}}\, \hat{\underline v})_2 \det\mathcal{B}^{inv}_{21}
+(\widehat{\mathop{\rm curl}}\, \hat{\underline v})_3 \det\mathcal{B}^{inv}_{31} \,,
$%
\lthtmlindisplaymathZ
\lthtmlcheckvsize\clearpage}

{\newpage\clearpage
\lthtmlinlinemathA{tex2html_wrap_inline6160}%
$ \mathcal{B}^{inv}_{ij}$%
\lthtmlinlinemathZ
\lthtmlcheckvsize\clearpage}

{\newpage\clearpage
\lthtmlinlinemathA{tex2html_wrap_inline6162}%
$ 2 \times 2$%
\lthtmlinlinemathZ
\lthtmlcheckvsize\clearpage}

{\newpage\clearpage
\lthtmlinlinemathA{tex2html_wrap_inline6164}%
$ B_K^{-1}$%
\lthtmlinlinemathZ
\lthtmlcheckvsize\clearpage}

{\newpage\clearpage
\lthtmlinlinemathA{tex2html_wrap_inline6166}%
$ A \in \mathbb{R}^{3\times 3}$%
\lthtmlinlinemathZ
\lthtmlcheckvsize\clearpage}

{\newpage\clearpage
\lthtmlinlinemathA{tex2html_wrap_indisplay6168}%
$\displaystyle (A^{-1})_{ij} = \frac{1}{\det A} (-1)^{i+j} \det \mathcal{A}_{ji} \,,$%
\lthtmlindisplaymathZ
\lthtmlcheckvsize\clearpage}

{\newpage\clearpage
\lthtmlinlinemathA{tex2html_wrap_inline6170}%
$ \mathcal{A}_{ij}$%
\lthtmlinlinemathZ
\lthtmlcheckvsize\clearpage}

{\newpage\clearpage
\lthtmlinlinemathA{tex2html_wrap_inline6174}%
$ A$%
\lthtmlinlinemathZ
\lthtmlcheckvsize\clearpage}

{\newpage\clearpage
\lthtmlinlinemathA{tex2html_wrap_inline6178}%
$ A = B_K^{-1}$%
\lthtmlinlinemathZ
\lthtmlcheckvsize\clearpage}

{\newpage\clearpage
\lthtmlinlinemathA{tex2html_wrap_indisplay6180}%
$\displaystyle \frac{1}{\det B_K}\,\frac{1}{\det B_K^{-1}} (-1)^{1+j} \det \mathcal{B}^{inv}_{j1} (\widehat{\mathop{\rm curl}}\, \hat{\underline v})_j
= (\widehat{\mathop{\rm curl}}\, \hat{\underline v})_1 \det \mathcal{B}^{inv}_{11} 
-(\widehat{\mathop{\rm curl}}\, \hat{\underline v})_2 \det \mathcal{B}^{inv}_{21}
+(\widehat{\mathop{\rm curl}}\, \hat{\underline v})_3 \det \mathcal{B}^{inv}_{31} = \det  \mathrm{M_1}\,.
$%
\lthtmlindisplaymathZ
\lthtmlcheckvsize\clearpage}

\stepcounter{subsection}
{\newpage\clearpage
\lthtmlinlinemathA{tex2html_wrap_inline6183}%
$ C$%
\lthtmlinlinemathZ
\lthtmlcheckvsize\clearpage}

{\newpage\clearpage
\lthtmlinlinemathA{tex2html_wrap_inline6185}%
$ \hat{C} = [0,1]^d$%
\lthtmlinlinemathZ
\lthtmlcheckvsize\clearpage}

{\newpage\clearpage
\lthtmlinlinemathA{tex2html_wrap_indisplay6187}%
$\displaystyle C = F_C(\hat{C}) \quad C \ni x = B_C \hat{x} + \underline b_C \,, \hat{x} \in \hat{C}\,.
$%
\lthtmlindisplaymathZ
\lthtmlcheckvsize\clearpage}

\stepcounter{subsubsection}
{\newpage\clearpage
\lthtmlinlinemathA{tex2html_wrap_inline6191}%
$ \mathcal{Q}_{l,m}$%
\lthtmlinlinemathZ
\lthtmlcheckvsize\clearpage}

{\newpage\clearpage
\lthtmlinlinemathA{tex2html_wrap_inline6193}%
$ \hat{C}$%
\lthtmlinlinemathZ
\lthtmlcheckvsize\clearpage}

{\newpage\clearpage
\lthtmlinlinemathA{tex2html_wrap_inline6195}%
$ l$%
\lthtmlinlinemathZ
\lthtmlcheckvsize\clearpage}

{\newpage\clearpage
\lthtmlinlinemathA{tex2html_wrap_inline6197}%
$ \hat{x}_1$%
\lthtmlinlinemathZ
\lthtmlcheckvsize\clearpage}

{\newpage\clearpage
\lthtmlinlinemathA{tex2html_wrap_inline6199}%
$ m$%
\lthtmlinlinemathZ
\lthtmlcheckvsize\clearpage}

{\newpage\clearpage
\lthtmlinlinemathA{tex2html_wrap_inline6201}%
$ \hat{x}_2$%
\lthtmlinlinemathZ
\lthtmlcheckvsize\clearpage}

{\newpage\clearpage
\lthtmlinlinemathA{tex2html_wrap_inline6203}%
$ \mathcal{Q}_{l,m,n}$%
\lthtmlinlinemathZ
\lthtmlcheckvsize\clearpage}

{\newpage\clearpage
\lthtmlinlinemathA{tex2html_wrap_inline6217}%
$ \hat{x}_3$%
\lthtmlinlinemathZ
\lthtmlcheckvsize\clearpage}

{\newpage\clearpage
\lthtmlinlinemathA{tex2html_wrap_indisplay6221}%
$\displaystyle \mathcal{P}^k = \left\{ \hat{\underline u} =  			\left(\begin{array}{cc} \hat{u}_1 \\  \hat{u}_2 \end{array}\right): \quad  						 \hat{u}_1 \in \mathcal{Q}_{k-1,k}\,, 								\hat{u}_2 \in \mathcal{Q}_{k,k-1} \right\}\,,$%
\lthtmlindisplaymathZ
\lthtmlcheckvsize\clearpage}

{\newpage\clearpage
\lthtmlinlinemathA{tex2html_wrap_indisplay6223}%
$\displaystyle \mathcal{P}^k = \left\{ \hat{\underline u} = \left(\begin{array}{ccc}  					\hat{u}_1 \\  \hat{u}_2 \\ \hat{u}_3 \end{array}\right):\quad  								\hat{u}_1 \in \mathcal{Q}_{k-1,k,k}\,, 								\hat{u}_2 \in \mathcal{Q}_{k,k-1,k}\,,  								\hat{u}_3 \in \mathcal{Q}_{k,k,k-1}\right\}\,.$%
\lthtmlindisplaymathZ
\lthtmlcheckvsize\clearpage}

\stepcounter{subsubsection}
{\newpage\clearpage
\lthtmlinlinemathA{tex2html_wrap_inline6228}%
$ \hat{C}\subset \mathbb{R}^2$%
\lthtmlinlinemathZ
\lthtmlcheckvsize\clearpage}

{\newpage\clearpage
\lthtmlinlinemathA{tex2html_wrap_inline6237}%
$ \mathcal{P}^k$%
\lthtmlinlinemathZ
\lthtmlcheckvsize\clearpage}

{\newpage\clearpage
\lthtmlinlinemathA{tex2html_wrap_indisplay6241}%
$\displaystyle \hat{\alpha}(\hat{\underline u}) := \int_{\hat{e}} (\hat{\underline t}\cdot \hat{\underline u})\, \hat{\varphi }\,d\hat{s}\,, \quad  			\forall\, \hat{\varphi } \in \mathbb{P}_{k-1}(\hat{e})\,,$%
\lthtmlindisplaymathZ
\lthtmlcheckvsize\clearpage}

{\newpage\clearpage
\lthtmlinlinemathA{tex2html_wrap_inline6247}%
$ 4k$%
\lthtmlinlinemathZ
\lthtmlcheckvsize\clearpage}

{\newpage\clearpage
\lthtmlinlinemathA{tex2html_wrap_indisplay6249}%
$\displaystyle \hat{\alpha}(\hat{\underline u}) := \int_{\hat{C}}  \hat{\underline u}\cdot \hat{\underline \varphi }\,d\hat{x}\,, \quad  			\forall\, \hat{\underline \varphi } = \left(\begin{array}{cc} \hat{\varphi }_1 \\  \hat{\varphi }_2 \end{array}\right) \,, 			\quad\hat{\varphi }_1\in\mathcal{Q}_{k-2,k-1}\,,\quad\hat{\varphi }_2\in\mathcal{Q}_{k-1,k-2}\,.$%
\lthtmlindisplaymathZ
\lthtmlcheckvsize\clearpage}

{\newpage\clearpage
\lthtmlinlinemathA{tex2html_wrap_inline6251}%
$ 2k(k-1)$%
\lthtmlinlinemathZ
\lthtmlcheckvsize\clearpage}

{\newpage\clearpage
\lthtmlinlinemathA{tex2html_wrap_inline6273}%
$ 12k$%
\lthtmlinlinemathZ
\lthtmlcheckvsize\clearpage}

{\newpage\clearpage
\lthtmlinlinemathA{tex2html_wrap_indisplay6275}%
$\displaystyle \hat{\alpha}(\hat{\underline u}) := \int_{\hat{f}} (\hat{\underline u}\wedge \hat{\underline n})\cdot \hat{\underline \varphi }\,d\hat{a} \,,\quad 			\forall \,\hat{\underline \varphi } = \left(\begin{array}{cc} \hat{\varphi }_1 \\  \hat{\varphi }_2 \end{array}\right) \,, 			\quad\hat{\varphi }_1\in\mathcal{Q}_{k-2,k-1}(\hat{f})\,,\quad\hat{\varphi }_2\in\mathcal{Q}_{k-1,k-2}(\hat{f})\,.$%
\lthtmlindisplaymathZ
\lthtmlcheckvsize\clearpage}

{\newpage\clearpage
\lthtmlinlinemathA{tex2html_wrap_inline6281}%
$ 6\cdot 2k(k-1)$%
\lthtmlinlinemathZ
\lthtmlcheckvsize\clearpage}

{\newpage\clearpage
\lthtmlinlinemathA{tex2html_wrap_indisplay6283}%
$\displaystyle \hat{\alpha}(\hat{\underline u}) := \int_{\hat{C}}  \hat{\underline u}\cdot \hat{\underline \varphi }\,d\hat{x} \,,\quad  			\forall\, \hat{\underline \varphi } = \left(\begin{array}{ccc} \hat{\varphi }_1 \\  \hat{\varphi }_2 \\   			\hat{\varphi }_3\end{array}\right) \,,\quad\hat{\varphi }_1\in\mathcal{Q}_{k-1,k-2,k-2}\,,\quad\hat{\varphi _2}\in\mathcal{Q}_{k-2,k-1,k-2}\,, 			\quad\hat{\varphi _3}\in\mathcal{Q}_{k-2,k-2,k-1}\,.$%
\lthtmlindisplaymathZ
\lthtmlcheckvsize\clearpage}

{\newpage\clearpage
\lthtmlinlinemathA{tex2html_wrap_inline6285}%
$ 3k(k-1)^2$%
\lthtmlinlinemathZ
\lthtmlcheckvsize\clearpage}

{\newpage\clearpage
\lthtmlinlinemathA{tex2html_wrap_inline6288}%
$ [0,1]^2$%
\lthtmlinlinemathZ
\lthtmlcheckvsize\clearpage}

{\newpage\clearpage
\lthtmlinlinemathA{tex2html_wrap_indisplay6290}%
$\displaystyle \hat{\underline t}_0 = \left(\begin{array}{cc} 1 \\  0 \end{array}\right)\,,\quad
\hat{\underline t}_1 = \left(\begin{array}{cc} 0 \\  1 \end{array}\right)\,,\quad
\hat{\underline t}_2 = \left(\begin{array}{cc} -1 \\  0 \end{array}\right)\,, \quad
\hat{\underline t}_3 = \left(\begin{array}{cc} 0 \\  -1 \end{array}\right)\,,
$%
\lthtmlindisplaymathZ
\lthtmlcheckvsize\clearpage}

{\newpage\clearpage
\lthtmlinlinemathA{tex2html_wrap_indisplay6292}%
$\displaystyle \hat{\underline N}_0 = \left(\begin{array}{cc} 1-\hat{y} \\  0 \end{array}\right)\,,\quad 		\hat{\underline N}_1 = \left(\begin{array}{cc} 0 \\  \hat{x} \end{array}\right)\,,\quad 		\hat{\underline N}_2 = \left(\begin{array}{cc} -\hat{y} \\  0 \end{array}\right)\,,\quad 		\hat{\underline N}_3 = \left(\begin{array}{cc} 0 \\  \hat{x}-1 \end{array}\right)\,.$%
\lthtmlindisplaymathZ
\lthtmlcheckvsize\clearpage}

\stepcounter{subsubsection}
\stepcounter{subsection}
{\newpage\clearpage
\lthtmlinlinemathA{tex2html_wrap_inline6296}%
$ F_C(\hat{C})$%
\lthtmlinlinemathZ
\lthtmlcheckvsize\clearpage}

{\newpage\clearpage
\lthtmlinlinemathA{tex2html_wrap_inline6300}%
$ \hat{D}F_C(\hat{x})$%
\lthtmlinlinemathZ
\lthtmlcheckvsize\clearpage}

{\newpage\clearpage
\lthtmlinlinemathA{tex2html_wrap_inline6302}%
$ F_C$%
\lthtmlinlinemathZ
\lthtmlcheckvsize\clearpage}

\stepcounter{subsubsection}
{\newpage\clearpage
\lthtmlinlinemathA{tex2html_wrap_indisplay6307}%
$\displaystyle \underline v(x) = (\hat{D}F_C^{-T} \hat{\underline v}_i) \circ F_C^{-1} (x)$%
\lthtmlindisplaymathZ
\lthtmlcheckvsize\clearpage}

{\newpage\clearpage
\lthtmlinlinemathA{tex2html_wrap_inline6315}%
$ \mathop{\rm curl}\underline v$%
\lthtmlinlinemathZ
\lthtmlcheckvsize\clearpage}

{\newpage\clearpage
\lthtmlinlinemathA{tex2html_wrap_inline6317}%
$ \widehat{\mathop{\rm curl}}\,\hat{\underline v}$%
\lthtmlinlinemathZ
\lthtmlcheckvsize\clearpage}

{\newpage\clearpage
\lthtmlinlinemathA{tex2html_wrap_indisplay6330}%
$\displaystyle \mathop{\rm curl}\underline v(x) = (\det \hat{D}F)^{-1} \widehat{\mathop{\rm curl}}\, \hat{\underline v}(\hat{x})\,, \qquad  x = F(\hat{x})\,,$%
\lthtmlindisplaymathZ
\lthtmlcheckvsize\clearpage}

{\newpage\clearpage
\lthtmlinlinemathA{tex2html_wrap_inline6334}%
$ F$%
\lthtmlinlinemathZ
\lthtmlcheckvsize\clearpage}

{\newpage\clearpage
\lthtmlinlinemathA{tex2html_wrap_inline6338}%
$ (\hat{D}F(F^{-1}(x)))^{-1} = D(F^{-1})(x)$%
\lthtmlinlinemathZ
\lthtmlcheckvsize\clearpage}

{\newpage\clearpage
\lthtmlinlinemathA{tex2html_wrap_inline6340}%
$ D(F^{-1})_{ij}(x)=
\frac{\partial \hat{x}_i}{\partial x_j}(x)$%
\lthtmlinlinemathZ
\lthtmlcheckvsize\clearpage}

{\newpage\clearpage
\lthtmlinlinemathA{tex2html_wrap_indisplay6342}%
$\displaystyle v_i(x) = \frac{\partial \hat{x}_j}{\partial x_i}(x) \,\hat{\underline v}_j (F^{-1}(x)) \,, \qquad i=1,2\,.
$%
\lthtmlindisplaymathZ
\lthtmlcheckvsize\clearpage}

{\newpage\clearpage
\lthtmldisplayA{displaymath6344}%
\begin{displaymath}\begin{split} 			 \frac{\partial v_2}{\partial x_1} &= \frac{\partial \hat{x}_i}{\partial x_2}(x) \,\frac{\partial }{\partial x_1}\hat{\underline v}_i(F^{-1}(x)) \\  			 \frac{\partial v_1}{\partial x_2} &= \frac{\partial \hat{x}_i}{\partial x_1}(x) \,\frac{\partial }{\partial x_2}\hat{\underline v}_i(F^{-1}(x)) \,, 			\end{split}\end{displaymath}%
\lthtmldisplayZ
\lthtmlcheckvsize\clearpage}

{\newpage\clearpage
\lthtmldisplayA{displaymath6346}%
\begin{displaymath}\begin{split} 			 \frac{\partial v_2}{\partial x_1} &= \frac{\partial ^2 \hat{x}_i}{\partial x_1\partial x_2}(x)\, \hat{\underline v}_i(F^{-1}(x)) +  			 						  \frac{\partial \hat{x}_i}{\partial x_2}(x) \,\frac{\partial }{\partial x_1}\hat{\underline v}_i(F^{-1}(x)) \\  			 \frac{\partial v_1}{\partial x_2} &= \frac{\partial ^2 \hat{x}_i}{\partial x_1\partial x_2}(x) \,\hat{\underline v}_i(F^{-1}(x)) + 			 						  \frac{\partial \hat{x}_i}{\partial x_1}(x) \,\frac{\partial }{\partial x_2}\hat{\underline v}_i(F^{-1}(x)) \,. 			\end{split}\end{displaymath}%
\lthtmldisplayZ
\lthtmlcheckvsize\clearpage}

{\newpage\clearpage
\lthtmlinlinemathA{tex2html_wrap_indisplay6348}%
$\displaystyle \mathop{\rm curl}\underline v = \frac{\partial v_2}{\partial x_1} - \frac{\partial v_1}{\partial x_2} = 
\frac{\partial \hat{x}_i}{\partial x_2}(x) \,\frac{\partial }{\partial x_1}\hat{\underline v}_i(F^{-1}(x)) 
- \frac{\partial \hat{x}_i}{\partial x_1}(x) \,\frac{\partial }{\partial x_2}\hat{\underline v}_i(F^{-1}(x)) \,
$%
\lthtmlindisplaymathZ
\lthtmlcheckvsize\clearpage}

\stepcounter{subsubsection}
{\newpage\clearpage
\lthtmlinlinemathA{tex2html_wrap_inline6353}%
$ D(F_C^{-1})(x)$%
\lthtmlinlinemathZ
\lthtmlcheckvsize\clearpage}

{\newpage\clearpage
\lthtmlinlinemathA{tex2html_wrap_inline6355}%
$ \frac{\partial v_i}{\partial x_j} - \frac{\partial v_j}{\partial x_i}$%
\lthtmlinlinemathZ
\lthtmlcheckvsize\clearpage}

{\newpage\clearpage
\lthtmlinlinemathA{tex2html_wrap_inline6357}%
$ i,j = 1,2,3$%
\lthtmlinlinemathZ
\lthtmlcheckvsize\clearpage}

{\newpage\clearpage
\lthtmlinlinemathA{tex2html_wrap_inline6359}%
$ \mathop{\rm Curl}v = Dv^T - Dv$%
\lthtmlinlinemathZ
\lthtmlcheckvsize\clearpage}

{\newpage\clearpage
\lthtmlinlinemathA{tex2html_wrap_indisplay6361}%
$\displaystyle \mathop{\rm Curl}v (x) = ((\hat{D}F_C^{-T} \widehat{\mathop{\rm Curl}}\,\hat{v} \,\hat{D}F_C^{-1}) \circ F_C^{-1})(x)
= (DF_C^{-1})^T(x)\, (\widehat{\mathop{\rm Curl}}\,\hat{v}\circ F_C^{-1})(x)\,DF_C^{-1}(x) \,.
$%
\lthtmlindisplaymathZ
\lthtmlcheckvsize\clearpage}

{\newpage\clearpage
\lthtmlinlinemathA{tex2html_wrap_inline6363}%
$ B_C$%
\lthtmlinlinemathZ
\lthtmlcheckvsize\clearpage}

{\newpage\clearpage
\lthtmlinlinemathA{tex2html_wrap_inline6365}%
$ \hat{D}F_C(\hat(x))$%
\lthtmlinlinemathZ
\lthtmlcheckvsize\clearpage}

{\newpage\clearpage
\lthtmlinlinemathA{tex2html_wrap_inline6370}%
$ C = F_C(\hat{C})$%
\lthtmlinlinemathZ
\lthtmlcheckvsize\clearpage}

{\newpage\clearpage
\lthtmlinlinemathA{tex2html_wrap_indisplay6376}%
$\displaystyle \mathop{\rm curl}\underline v =  				\left(\frac{1}{\det \hat{D}F_C}\,\hat{D}F_C\, \widehat{\mathop{\rm curl}}\,\hat{\underline v} \right) \circ F_C^{-1}\,.$%
\lthtmlindisplaymathZ
\lthtmlcheckvsize\clearpage}

\stepcounter{subsection}
{\newpage\clearpage
\lthtmlinlinemathA{tex2html_wrap_inline6383}%
$ \mathcal{Q}^k$%
\lthtmlinlinemathZ
\lthtmlcheckvsize\clearpage}

{\newpage\clearpage
\lthtmlinlinemathA{tex2html_wrap_inline6389}%
$ \mathcal{P}_K$%
\lthtmlinlinemathZ
\lthtmlcheckvsize\clearpage}

{\newpage\clearpage
\lthtmlinlinemathA{tex2html_wrap_inline6393}%
$ K=F(\hat{K})$%
\lthtmlinlinemathZ
\lthtmlcheckvsize\clearpage}

{\newpage\clearpage
\lthtmlinlinemathA{tex2html_wrap_inline6399}%
$ [0,|e|] \ni s \mapsto \underline x(s) \in e$%
\lthtmlinlinemathZ
\lthtmlcheckvsize\clearpage}

{\newpage\clearpage
\lthtmlinlinemathA{tex2html_wrap_inline6401}%
$ [0,|\hat{e}|] \ni \hat{s} \mapsto \hat{\underline x}(\hat{s}) \in \hat{e}$%
\lthtmlinlinemathZ
\lthtmlcheckvsize\clearpage}

{\newpage\clearpage
\lthtmlinlinemathA{tex2html_wrap_inline6411}%
$ \frac{d \underline x}{ds}$%
\lthtmlinlinemathZ
\lthtmlcheckvsize\clearpage}

{\newpage\clearpage
\lthtmlinlinemathA{tex2html_wrap_inline6413}%
$ \frac{d \hat{\underline x}}{d\hat{s}}$%
\lthtmlinlinemathZ
\lthtmlcheckvsize\clearpage}

{\newpage\clearpage
\lthtmlinlinemathA{tex2html_wrap_indisplay6422}%
$\displaystyle \underline v\cdot \underline t = \frac{|\hat{e}|}{|e|} (\hat{\underline v}\cdot \hat{\underline t})\,,$%
\lthtmlindisplaymathZ
\lthtmlcheckvsize\clearpage}

{\newpage\clearpage
\lthtmlinlinemathA{tex2html_wrap_inline6424}%
$ |\hat{e}|$%
\lthtmlinlinemathZ
\lthtmlcheckvsize\clearpage}

{\newpage\clearpage
\lthtmlinlinemathA{tex2html_wrap_inline6426}%
$ |e|$%
\lthtmlinlinemathZ
\lthtmlcheckvsize\clearpage}

{\newpage\clearpage
\lthtmlinlinemathA{tex2html_wrap_indisplay6432}%
$\displaystyle (\underline v(x))_i = (D(F^{-1})^T \hat{\underline v})_i = \frac{\partial \hat{x}_j}{\partial x_i}(x) \hat{\underline v}_j(\hat{x})\,
$%
\lthtmlindisplaymathZ
\lthtmlcheckvsize\clearpage}

{\newpage\clearpage
\lthtmlinlinemathA{tex2html_wrap_inline6434}%
$ \hat{x}_j = \hat{x}_j(\underline x(s))$%
\lthtmlinlinemathZ
\lthtmlcheckvsize\clearpage}

{\newpage\clearpage
\lthtmlinlinemathA{tex2html_wrap_inline6436}%
$ \hat{x}_j = \hat{x}_j(\hat{s}(s))$%
\lthtmlinlinemathZ
\lthtmlcheckvsize\clearpage}

{\newpage\clearpage
\lthtmlinlinemathA{tex2html_wrap_indisplay6438}%
$\displaystyle \underline v\cdot \underline t= \underline v \cdot \frac{d \underline x}{ds} = \left( \hat{\underline v}_j\frac{\partial \hat{x}_j}{\partial x_i}\right) (x)\frac{dx_i}{ds}
= \hat{\underline v}_j \frac{d \hat{x}_j}{ds} = \hat{\underline v}_j \frac{d \hat{x}_j}{d\hat{s}} \frac{d\hat{s}}{ds} 
= (\hat{\underline v}\cdot \hat{\underline t}) \frac{d\hat{s}}{ds} 
$%
\lthtmlindisplaymathZ
\lthtmlcheckvsize\clearpage}

{\newpage\clearpage
\lthtmlinlinemathA{tex2html_wrap_inline6440}%
$ \frac{d\hat{s}}{ds}=\frac{|\hat{e}|}{|e|}$%
\lthtmlinlinemathZ
\lthtmlcheckvsize\clearpage}

{\newpage\clearpage
\lthtmlinlinemathA{tex2html_wrap_inline6451}%
$ \alpha^{[K]}(\underline u) := \int_e (\underline v\cdot \underline t)\varphi \,ds$%
\lthtmlinlinemathZ
\lthtmlcheckvsize\clearpage}

{\newpage\clearpage
\lthtmlinlinemathA{tex2html_wrap_indisplay6453}%
$\displaystyle \alpha^{[K]}(\underline u) = \int_e (\underline v\cdot \underline t)\varphi \,ds = 
\int_{\hat{e}} (\hat{\underline v} \cdot \hat{\underline t}) \hat{\varphi } \, d\hat{s}\, = \hat{\alpha}(\hat{\underline u})\,,
\qquad \forall\, \hat{\varphi } \in
\mathbb{P}_{k-1}(\hat{e})\,, \quad \varphi = \hat{\varphi } \circ F^{-1} \in \mathbb{P}_{k-1}(e)\,.
$%
\lthtmlindisplaymathZ
\lthtmlcheckvsize\clearpage}

{\newpage\clearpage
\lthtmlinlinemathA{tex2html_wrap_inline6455}%
$ K_- = F_-(\hat{K})$%
\lthtmlinlinemathZ
\lthtmlcheckvsize\clearpage}

{\newpage\clearpage
\lthtmlinlinemathA{tex2html_wrap_inline6457}%
$ K_+ = F_+(\hat{K})$%
\lthtmlinlinemathZ
\lthtmlcheckvsize\clearpage}

{\newpage\clearpage
\lthtmlinlinemathA{tex2html_wrap_inline6461}%
$ \underline N$%
\lthtmlinlinemathZ
\lthtmlcheckvsize\clearpage}

{\newpage\clearpage
\lthtmlinlinemathA{tex2html_wrap_inline6465}%
$ \underline N_-$%
\lthtmlinlinemathZ
\lthtmlcheckvsize\clearpage}

{\newpage\clearpage
\lthtmlinlinemathA{tex2html_wrap_inline6467}%
$ \underline N_+$%
\lthtmlinlinemathZ
\lthtmlcheckvsize\clearpage}

{\newpage\clearpage
\lthtmlinlinemathA{tex2html_wrap_inline6475}%
$ e_+ =F_+(\hat{e}_i)$%
\lthtmlinlinemathZ
\lthtmlcheckvsize\clearpage}

{\newpage\clearpage
\lthtmlinlinemathA{tex2html_wrap_inline6477}%
$ e_- =F_-(\hat{e}_j)$%
\lthtmlinlinemathZ
\lthtmlcheckvsize\clearpage}

{\newpage\clearpage
\lthtmlinlinemathA{tex2html_wrap_inline6479}%
$ \underline t_+ $%
\lthtmlinlinemathZ
\lthtmlcheckvsize\clearpage}

{\newpage\clearpage
\lthtmlinlinemathA{tex2html_wrap_inline6485}%
$ \underline t_- = -\underline t_+$%
\lthtmlinlinemathZ
\lthtmlcheckvsize\clearpage}

{\newpage\clearpage
\lthtmlinlinemathA{tex2html_wrap_inline6491}%
$ \int_{e_+}$%
\lthtmlinlinemathZ
\lthtmlcheckvsize\clearpage}

{\newpage\clearpage
\lthtmlinlinemathA{tex2html_wrap_inline6495}%
$ \int_{e_-}$%
\lthtmlinlinemathZ
\lthtmlcheckvsize\clearpage}

{\newpage\clearpage
\lthtmlinlinemathA{tex2html_wrap_inline6499}%
$ \underline t_-$%
\lthtmlinlinemathZ
\lthtmlcheckvsize\clearpage}

{\newpage\clearpage
\lthtmlinlinemathA{tex2html_wrap_indisplay6503}%
$\displaystyle \underline N_+\cdot\underline t_+ + \underline N_-\cdot\underline t_- = 0\,.$%
\lthtmlindisplaymathZ
\lthtmlcheckvsize\clearpage}

{\newpage\clearpage
\lthtmlinlinemathA{tex2html_wrap_inline6505}%
$ \alpha^{[K_+]}$%
\lthtmlinlinemathZ
\lthtmlcheckvsize\clearpage}

{\newpage\clearpage
\lthtmlinlinemathA{tex2html_wrap_inline6507}%
$ \alpha^{[K_-]}$%
\lthtmlinlinemathZ
\lthtmlcheckvsize\clearpage}

{\newpage\clearpage
\lthtmlinlinemathA{tex2html_wrap_inline6514}%
$ \hat{e} \ni \hat{x}(s) := \underline a +
s\, \hat{\underline t}$%
\lthtmlinlinemathZ
\lthtmlcheckvsize\clearpage}

{\newpage\clearpage
\lthtmlinlinemathA{tex2html_wrap_inline6516}%
$ \hat{\underline p} \in \mathcal{S}^k$%
\lthtmlinlinemathZ
\lthtmlcheckvsize\clearpage}

{\newpage\clearpage
\lthtmlinlinemathA{tex2html_wrap_indisplay6520}%
$\displaystyle (\hat{\underline p}\cdot \hat{\underline t})|_{\hat{e}} \in \mathbb{P}_{k-1}(\hat{e})\,.
$%
\lthtmlindisplaymathZ
\lthtmlcheckvsize\clearpage}

{\newpage\clearpage
\lthtmlinlinemathA{tex2html_wrap_indisplay6522}%
$\displaystyle \hat{\underline p} \in \mathcal{S}^k \quad \Longrightarrow \quad \textrm{for } i=1,2,3: 
\quad \hat{p}_i(\hat{x}) = \prod_{j=1}^3 \hat{x}_j^{k_{ij}}\,, \quad
\textrm{where } \sum_{j=1}^3 k_{ij} = k\,.
$%
\lthtmlindisplaymathZ
\lthtmlcheckvsize\clearpage}

{\newpage\clearpage
\lthtmlinlinemathA{tex2html_wrap_inline6526}%
$ \hat{x}(s)$%
\lthtmlinlinemathZ
\lthtmlcheckvsize\clearpage}

{\newpage\clearpage
\lthtmlinlinemathA{tex2html_wrap_indisplay6528}%
$\displaystyle \hat{p}_i(\hat{x}(s)) = \prod_{j=1}^3 (a_j + s\,\hat{t}_j)^{k_{ij}} = s^k\,\prod_{j=1}^3 \hat{t}_j^{k_{ij}} + 
\hat{\varphi }_{k-1}(s)\,,
$%
\lthtmlindisplaymathZ
\lthtmlcheckvsize\clearpage}

{\newpage\clearpage
\lthtmlinlinemathA{tex2html_wrap_inline6530}%
$ \hat{\varphi }_{k-1}(s) \in \mathbb{P}_{k-1}(\hat{e})$%
\lthtmlinlinemathZ
\lthtmlcheckvsize\clearpage}

{\newpage\clearpage
\lthtmlinlinemathA{tex2html_wrap_indisplay6532}%
$\displaystyle (\hat{\underline p}\cdot \hat{\underline t})|_{\hat{e}} = s^k\,\sum_{i=1}^3\hat{t}_i\left(\prod_{j=1}^3 \hat{t}_j^{k_{ij}}\right) +
\hat{\varphi }_{k-1}(s)\,.
$%
\lthtmlindisplaymathZ
\lthtmlcheckvsize\clearpage}

{\newpage\clearpage
\lthtmlinlinemathA{tex2html_wrap_inline6534}%
$ s^k$%
\lthtmlinlinemathZ
\lthtmlcheckvsize\clearpage}

{\newpage\clearpage
\lthtmlinlinemathA{tex2html_wrap_inline6536}%
$ \hat{\underline p}(\hat{\underline t}) \cdot \hat{\underline t}$%
\lthtmlinlinemathZ
\lthtmlcheckvsize\clearpage}

{\newpage\clearpage
\lthtmlinlinemathA{tex2html_wrap_inline6543}%
$ \hat{R} = \mathcal{P}^k$%
\lthtmlinlinemathZ
\lthtmlcheckvsize\clearpage}

{\newpage\clearpage
\lthtmlinlinemathA{tex2html_wrap_inline6547}%
$ (\hat{\underline v}\cdot \hat{\underline t})|_{\hat{e}} \in \mathbb{P}_{k-1}(\hat{e})$%
\lthtmlinlinemathZ
\lthtmlcheckvsize\clearpage}

{\newpage\clearpage
\lthtmlinlinemathA{tex2html_wrap_indisplay6560}%
$\displaystyle \underline N_+ := \mathcal{P}_+(\hat{\underline N}_i) = \hat{D}F_+^{-T} \hat{\underline N}_i \,,  			\qquad \underline N_- := -\,\mathcal{P}_-(\hat{\underline N}_j) = -\hat{D}F_-^{-T}\hat{\underline N}_j\,.$%
\lthtmlindisplaymathZ
\lthtmlcheckvsize\clearpage}

{\newpage\clearpage
\lthtmlinlinemathA{tex2html_wrap_inline6566}%
$ \underline v := \mathcal{P}_K(\hat{\underline v})$%
\lthtmlinlinemathZ
\lthtmlcheckvsize\clearpage}

{\newpage\clearpage
\lthtmlinlinemathA{tex2html_wrap_inline6570}%
$ \hat{\underline v} \in \hat{R}$%
\lthtmlinlinemathZ
\lthtmlcheckvsize\clearpage}

{\newpage\clearpage
\lthtmlinlinemathA{tex2html_wrap_inline6584}%
$ (\underline v\cdot \underline t)|_{e} \in \mathbb{P}_{k-1}(e)$%
\lthtmlinlinemathZ
\lthtmlcheckvsize\clearpage}

{\newpage\clearpage
\lthtmlinlinemathA{tex2html_wrap_indisplay6590}%
$\displaystyle \int_{e_+} \left((\underline N_+\cdot \underline t_+) + (\underline N_-\cdot \underline t_-)\right)\,\varphi \,ds\,, \qquad \forall\,\varphi \in\mathbb{P}_{k-1}(e)
$%
\lthtmlindisplaymathZ
\lthtmlcheckvsize\clearpage}

{\newpage\clearpage
\lthtmlinlinemathA{tex2html_wrap_inline6592}%
$ \int_{e_+}(\underline N_-\cdot \underline t_-)\,\varphi \,ds = -\int_{e_-}(\underline N_-\cdot \underline t_-)\,\varphi \,ds$%
\lthtmlinlinemathZ
\lthtmlcheckvsize\clearpage}

{\newpage\clearpage
\lthtmlinlinemathA{tex2html_wrap_indisplay6598}%
$\displaystyle \int_{e_+} (\underline N_+ \cdot \underline t_+)\varphi \,ds = \int_{\hat{e}_i}  (\hat{\underline N}_i\cdot \hat{\underline t}_i)\hat{\varphi }\,d\hat{s} = 1 			\qquad \textrm{and} \qquad 			\int_{e_-} (\underline N_- \cdot \underline t_-)\varphi \,ds = -\int_{\hat{e}_j}  (\hat{\underline N}_j\cdot \hat{\underline t}_j)\hat{\varphi }\,d\hat{s} = -1\,.$%
\lthtmlindisplaymathZ
\lthtmlcheckvsize\clearpage}

{\newpage\clearpage
\lthtmlinlinemathA{tex2html_wrap_inline6602}%
$ \hat{\alpha}_j(\hat{\underline 
v}) = \int_{\hat{e}_j} \hat{\underline v} \cdot \hat{\underline t}_j\,d\hat{s}$%
\lthtmlinlinemathZ
\lthtmlcheckvsize\clearpage}

{\newpage\clearpage
\lthtmlinlinemathA{tex2html_wrap_inline6604}%
$ \hat{\alpha}_j(\hat{\underline N}_i) = \delta_{ij}$%
\lthtmlinlinemathZ
\lthtmlcheckvsize\clearpage}

{\newpage\clearpage
\lthtmlinlinemathA{tex2html_wrap_indisplay6606}%
$\displaystyle v_j = \hat{\alpha}_j(\hat{\underline v}) = (\hat{\underline N}_j \cdot \hat{\underline t})\,|\hat{e}_j| = ({\underline N}_j \cdot {\underline t}_j)\, |e_j|\,,
$%
\lthtmlindisplaymathZ
\lthtmlcheckvsize\clearpage}

{\newpage\clearpage
\lthtmlinlinemathA{tex2html_wrap_inline6608}%
$ \alpha_j(\underline v)$%
\lthtmlinlinemathZ
\lthtmlcheckvsize\clearpage}

{\newpage\clearpage
\lthtmlinlinemathA{tex2html_wrap_inline6610}%
$ e_j$%
\lthtmlinlinemathZ
\lthtmlcheckvsize\clearpage}

{\newpage\clearpage
\lthtmlinlinemathA{tex2html_wrap_inline6612}%
$ |e_j|\left(\underline v\cdot \underline t_j\right)|_e$%
\lthtmlinlinemathZ
\lthtmlcheckvsize\clearpage}

{\newpage\clearpage
\lthtmlinlinemathA{tex2html_wrap_inline6615}%
$ \alpha^{[K]}$%
\lthtmlinlinemathZ
\lthtmlcheckvsize\clearpage}

{\newpage\clearpage
\lthtmlinlinemathA{tex2html_wrap_inline6619}%
$ \underline t = \frac{|\hat{e}|}{|e|}\,(\hat{D}F)\,\hat{\underline t}$%
\lthtmlinlinemathZ
\lthtmlcheckvsize\clearpage}

{\newpage\clearpage
\lthtmlinlinemathA{tex2html_wrap_inline6623}%
$ \tilde{\underline t} = (\hat{D}F)\,\hat{\underline t}$%
\lthtmlinlinemathZ
\lthtmlcheckvsize\clearpage}

\stepcounter{subsection}
{\newpage\clearpage
\lthtmlinlinemathA{tex2html_wrap_inline6626}%
$ V_h \subset H(\mathop{\rm curl};\Omega )$%
\lthtmlinlinemathZ
\lthtmlcheckvsize\clearpage}

{\newpage\clearpage
\lthtmlinlinemathA{tex2html_wrap_indisplay6628}%
$\displaystyle \| \underline u - \Pi_h^k \underline u\|_{H(\mathop{\rm curl}; \Omega )} = C\,\inf_{w\in V_h}\| \underline u -  \underline w\|_{H(\mathop{\rm curl}; \Omega )}\,,
$%
\lthtmlindisplaymathZ
\lthtmlcheckvsize\clearpage}

{\newpage\clearpage
\lthtmlinlinemathA{tex2html_wrap_inline6630}%
$ \Pi_h^k \underline u \in \mathcal{R}^k$%
\lthtmlinlinemathZ
\lthtmlcheckvsize\clearpage}

{\newpage\clearpage
\lthtmlinlinemathA{tex2html_wrap_inline6632}%
$ \Pi_h^k \underline u \in \mathcal{P}^k$%
\lthtmlinlinemathZ
\lthtmlcheckvsize\clearpage}

{\newpage\clearpage
\lthtmlinlinemathA{tex2html_wrap_inline6636}%
$ \alpha(\underline u) = \alpha(\Pi_h^k \underline u)$%
\lthtmlinlinemathZ
\lthtmlcheckvsize\clearpage}

{\newpage\clearpage
\lthtmlinlinemathA{tex2html_wrap_inline6638}%
$ \alpha$%
\lthtmlinlinemathZ
\lthtmlcheckvsize\clearpage}

{\newpage\clearpage
\lthtmlinlinemathA{tex2html_wrap_inline6640}%
$ \Pi_h^k$%
\lthtmlinlinemathZ
\lthtmlcheckvsize\clearpage}

{\newpage\clearpage
\lthtmlinlinemathA{tex2html_wrap_inline6642}%
$ \underline v\in H^r(\mathop{\rm curl})$%
\lthtmlinlinemathZ
\lthtmlcheckvsize\clearpage}

{\newpage\clearpage
\lthtmlinlinemathA{tex2html_wrap_inline6644}%
$ r>\frac{1}{2}$%
\lthtmlinlinemathZ
\lthtmlcheckvsize\clearpage}

{\newpage\clearpage
\lthtmlinlinemathA{tex2html_wrap_inline6647}%
$ \mathcal{T}_h$%
\lthtmlinlinemathZ
\lthtmlcheckvsize\clearpage}

{\newpage\clearpage
\lthtmlinlinemathA{tex2html_wrap_inline6649}%
$ h>0$%
\lthtmlinlinemathZ
\lthtmlcheckvsize\clearpage}

{\newpage\clearpage
\lthtmlinlinemathA{tex2html_wrap_inline6655}%
$ C>0$%
\lthtmlinlinemathZ
\lthtmlcheckvsize\clearpage}

{\newpage\clearpage
\lthtmlinlinemathA{tex2html_wrap_inline6657}%
$ r$%
\lthtmlinlinemathZ
\lthtmlcheckvsize\clearpage}

{\newpage\clearpage
\lthtmlinlinemathA{tex2html_wrap_inline6659}%
$ h$%
\lthtmlinlinemathZ
\lthtmlcheckvsize\clearpage}

{\newpage\clearpage
\lthtmlinlinemathA{tex2html_wrap_indisplay6663}%
$\displaystyle \| \underline v - \Pi_h^k \underline v\|_{H(\mathop{\rm curl}; \Omega )} \leq C\,h^{\min\{r,k\}} \|\underline v\|_{H^r(\mathop{\rm curl};\Omega )}\,,$%
\lthtmlindisplaymathZ
\lthtmlcheckvsize\clearpage}

{\newpage\clearpage
\lthtmlinlinemathA{tex2html_wrap_inline6665}%
$ \underline v\in H^r(\mathop{\rm curl};\Omega )$%
\lthtmlinlinemathZ
\lthtmlcheckvsize\clearpage}

{\newpage\clearpage
\lthtmlinlinemathA{tex2html_wrap_inline6675}%
$ \mathcal{O}(h^k)$%
\lthtmlinlinemathZ
\lthtmlcheckvsize\clearpage}

{\newpage\clearpage
\lthtmlinlinemathA{tex2html_wrap_inline6677}%
$ [\mathbb{P}^{k-1}(K)]^d \subseteq \mathcal{R}^k(K)$%
\lthtmlinlinemathZ
\lthtmlcheckvsize\clearpage}

{\newpage\clearpage
\lthtmlinlinemathA{tex2html_wrap_indisplay6679}%
$\displaystyle \| \underline v - \Pi_h^k \underline v\|_{L^2(\Omega )} \leq C h^k |\underline v|_{H^k(\Omega )}\,.$%
\lthtmlindisplaymathZ
\lthtmlcheckvsize\clearpage}

{\newpage\clearpage
\lthtmlinlinemathA{tex2html_wrap_inline6681}%
$ [\mathbb{P}^{k-1}(K)]^d \subseteq \mathcal{R}^k(K) \subsetneq [\mathbb{P}^k(K)]^d$%
\lthtmlinlinemathZ
\lthtmlcheckvsize\clearpage}

{\newpage\clearpage
\lthtmlinlinemathA{tex2html_wrap_inline6685}%
$ H^k(K)$%
\lthtmlinlinemathZ
\lthtmlcheckvsize\clearpage}

{\newpage\clearpage
\lthtmlinlinemathA{tex2html_wrap_inline6687}%
$ \mathcal{R}^k(K)$%
\lthtmlinlinemathZ
\lthtmlcheckvsize\clearpage}

{\newpage\clearpage
\lthtmlinlinemathA{tex2html_wrap_inline6691}%
$ \| \underline u - \underline u_h\|_{L^2(\Omega )}$%
\lthtmlinlinemathZ
\lthtmlcheckvsize\clearpage}

{\newpage\clearpage
\lthtmlinlinemathA{tex2html_wrap_inline6696}%
$ s>\frac{1}{2}$%
\lthtmlinlinemathZ
\lthtmlcheckvsize\clearpage}

{\newpage\clearpage
\lthtmlinlinemathA{tex2html_wrap_indisplay6698}%
$\displaystyle \| \underline u -  \underline u_h\|_{L^2(\Omega )} \leq C h^s \| \underline u - \underline u_h\|_{H(\mathop{\rm curl}; \Omega )}\,.$%
\lthtmlindisplaymathZ
\lthtmlcheckvsize\clearpage}

{\newpage\clearpage
\lthtmlinlinemathA{tex2html_wrap_inline6702}%
$ s=1$%
\lthtmlinlinemathZ
\lthtmlcheckvsize\clearpage}

{\newpage\clearpage
\lthtmlinlinemathA{tex2html_wrap_inline6704}%
$ [\mathbb{P}_k]^d$%
\lthtmlinlinemathZ
\lthtmlcheckvsize\clearpage}

\stepcounter{section}
{\newpage\clearpage
\lthtmlinlinemathA{tex2html_wrap_inline6711}%
$ \Omega = [-1,1]^d$%
\lthtmlinlinemathZ
\lthtmlcheckvsize\clearpage}

{\newpage\clearpage
\lthtmlinlinemathA{tex2html_wrap_indisplay6715}%
$\displaystyle c\equiv 1 \,, \qquad \underline f(x,y) = \left(\begin{array}{cc} 3 - y^2 \\  3 - x^2 \end{array}\right)\,.$%
\lthtmlindisplaymathZ
\lthtmlcheckvsize\clearpage}

{\newpage\clearpage
\lthtmlinlinemathA{tex2html_wrap_indisplay6717}%
$\displaystyle c\equiv 1 \,, \qquad  		\underline f(x,y) = (2\pi^2 + 1)\left(\begin{array}{cc} \cos\pi x\sin\pi y \\  -\sin\pi x\cos\pi y \end{array}\right)\,.$%
\lthtmlindisplaymathZ
\lthtmlcheckvsize\clearpage}

{\newpage\clearpage
\lthtmlinlinemathA{tex2html_wrap_inline6721}%
$ 2^5$%
\lthtmlinlinemathZ
\lthtmlcheckvsize\clearpage}

{\newpage\clearpage
\lthtmlinlinemathA{tex2html_wrap_inline6723}%
$ 2^{13}$%
\lthtmlinlinemathZ
\lthtmlcheckvsize\clearpage}

{\newpage\clearpage
\lthtmlinlinemathA{tex2html_wrap_inline6731}%
$ \mathcal{O}(h)$%
\lthtmlinlinemathZ
\lthtmlcheckvsize\clearpage}

{\newpage\clearpage
\lthtmlinlinemathA{tex2html_wrap_inline6733}%
$ L^2$%
\lthtmlinlinemathZ
\lthtmlcheckvsize\clearpage}

{\newpage\clearpage
\lthtmlinlinemathA{tex2html_wrap_inline6735}%
$ \#$%
\lthtmlinlinemathZ
\lthtmlcheckvsize\clearpage}

{\newpage\clearpage
\lthtmlinlinemathA{tex2html_wrap_inline6818}%
$ H(\mathop{\rm curl};\Omega )$%
\lthtmlinlinemathZ
\lthtmlcheckvsize\clearpage}

{\newpage\clearpage
\lthtmlpictureA{tex2html_wrap4265}%
% latex2html id marker 4265
\includegraphics[width=9.5cm, height=7cm]{example1_errors.eps}%
\lthtmlpictureZ
\lthtmlcheckvsize\clearpage}

{\newpage\clearpage
\lthtmlinlinemathA{tex2html_wrap_inline6830}%
$ H(\mathop{\rm curl};(\Omega ))$%
\lthtmlinlinemathZ
\lthtmlcheckvsize\clearpage}

{\newpage\clearpage
\lthtmlpictureA{tex2html_wrap4271}%
% latex2html id marker 4271
\includegraphics[width=9.5cm, height=7cm]{example2_errors.eps}%
\lthtmlpictureZ
\lthtmlcheckvsize\clearpage}

{\newpage\clearpage
\lthtmlpictureA{tex2html_wrap4293}%
\includegraphics[width=5.5cm, height=5.5cm]{grid.eps}%
\lthtmlpictureZ
\lthtmlcheckvsize\clearpage}

{\newpage\clearpage
\lthtmlinlinemathA{tex2html_wrap_indisplay6913}%
$\displaystyle c\equiv 1 \,, \qquad \underline f(x,y,z) = \left(\begin{array}{ccc}   xy(1 - y^2)(1-z^2) + 2xy(1-z^2) \\  																	  y^2(1 - x^2)(1-z^2) + (1-y^2)(2-x^2-z^2) \\  																	  yz(1 - x^2)(1-y^2) + 2yz(1-x^2) 											 \end{array}\right)\,.$%
\lthtmlindisplaymathZ
\lthtmlcheckvsize\clearpage}

{\newpage\clearpage
\lthtmlpictureA{tex2html_wrap4329}%
\includegraphics[width=9.5cm, height=7cm]{field1.eps}%
\lthtmlpictureZ
\lthtmlcheckvsize\clearpage}

{\newpage\clearpage
\lthtmlpictureA{tex2html_wrap4335}%
\includegraphics[width=9.5cm, height=7cm]{field2.eps}%
\lthtmlpictureZ
\lthtmlcheckvsize\clearpage}

{\newpage\clearpage
\lthtmlinlinemathA{tex2html_wrap_inline7106}%
$ \mathbb{R}^3$%
\lthtmlinlinemathZ
\lthtmlcheckvsize\clearpage}

\appendix
\stepcounter{section}
{\newpage\clearpage
\lthtmlinlinemathA{tex2html_wrap_inline7119}%
$ \mathbb{R}^2$%
\lthtmlinlinemathZ
\lthtmlcheckvsize\clearpage}

{\newpage\clearpage
\lthtmlinlinemathA{tex2html_wrap_inline7121}%
$ \varphi (x,y)$%
\lthtmlinlinemathZ
\lthtmlcheckvsize\clearpage}

{\newpage\clearpage
\lthtmlinlinemathA{tex2html_wrap_inline7125}%
$ c>0$%
\lthtmlinlinemathZ
\lthtmlcheckvsize\clearpage}

{\newpage\clearpage
\lthtmlinlinemathA{tex2html_wrap_inline7127}%
$ w$%
\lthtmlinlinemathZ
\lthtmlcheckvsize\clearpage}

{\newpage\clearpage
\lthtmldisplayA{displaymath7129}%
\begin{displaymath}\begin{split} 			-\Delta w + c\, w &= \varphi \quad \mathrm{in} \quad \Omega \\  			\underline n \cdot \nabla w &= 0 \quad \mathrm{on} \quad \partial \Omega \,. 		\end{split}\end{displaymath}%
\lthtmldisplayZ
\lthtmlcheckvsize\clearpage}

{\newpage\clearpage
\lthtmlinlinemathA{tex2html_wrap_inline7131}%
$ \underline E := \nabla^{\perp} w$%
\lthtmlinlinemathZ
\lthtmlcheckvsize\clearpage}

{\newpage\clearpage
\lthtmldisplayA{displaymath7133}%
\begin{displaymath}\begin{split} 			\underline \mathop{\rm curl}\mathop{\rm curl}\underline E + c\, \underline E  = \underline f \quad \mathrm{in} \quad \Omega \,, \\  			\underline E \wedge \underline n = 0 \quad \mathrm{on} \quad \partial \Omega \,, 		\end{split}\end{displaymath}%
\lthtmldisplayZ
\lthtmlcheckvsize\clearpage}

{\newpage\clearpage
\lthtmlinlinemathA{tex2html_wrap_inline7135}%
$ \underline f := \nabla^{\perp} \varphi $%
\lthtmlinlinemathZ
\lthtmlcheckvsize\clearpage}

{\newpage\clearpage
\lthtmlinlinemathA{tex2html_wrap_inline7137}%
$ \nabla^{\perp} \varphi := \boldsymbol{R}\nabla\varphi = \left(\begin{array}{cc} \partial _y\varphi \\  -\partial _x\varphi 
\end{array}\right)$%
\lthtmlinlinemathZ
\lthtmlcheckvsize\clearpage}

{\newpage\clearpage
\lthtmlinlinemathA{tex2html_wrap_indisplay7141}%
$\displaystyle \underline E \wedge \underline n = \underline E \cdot \underline t = {\nabla w}^T \boldsymbol{R}^T\boldsymbol{R}\, \underline n = \nabla w \cdot \underline n \,.
$%
\lthtmlindisplaymathZ
\lthtmlcheckvsize\clearpage}

{\newpage\clearpage
\lthtmlinlinemathA{tex2html_wrap_inline7143}%
$ \underline E$%
\lthtmlinlinemathZ
\lthtmlcheckvsize\clearpage}

{\newpage\clearpage
\lthtmlinlinemathA{tex2html_wrap_inline7147}%
$ \nabla\cdot\nabla^{\perp}w = 0$%
\lthtmlinlinemathZ
\lthtmlcheckvsize\clearpage}

{\newpage\clearpage
\lthtmlinlinemathA{tex2html_wrap_inline7151}%
$ \underline \mathop{\rm curl}\mathop{\rm curl}\underline E = \nabla(\nabla\cdot\underline E) - \Delta\underline E$%
\lthtmlinlinemathZ
\lthtmlcheckvsize\clearpage}

{\newpage\clearpage
\lthtmlinlinemathA{tex2html_wrap_inline7153}%
$ \underline \mathop{\rm curl}\mathop{\rm curl}\underline E = - \Delta\underline E$%
\lthtmlinlinemathZ
\lthtmlcheckvsize\clearpage}

{\newpage\clearpage
\lthtmlinlinemathA{tex2html_wrap_inline7155}%
$ \nabla^{\perp}w$%
\lthtmlinlinemathZ
\lthtmlcheckvsize\clearpage}

{\newpage\clearpage
\lthtmlinlinemathA{tex2html_wrap_inline7157}%
$ \nabla^{\perp} \varphi $%
\lthtmlinlinemathZ
\lthtmlcheckvsize\clearpage}

{\newpage\clearpage
\lthtmldisplayA{displaymath7162}%
\begin{displaymath}\begin{split} 			-\Delta w  &= \lambda \, w \quad \mathrm{in} \quad \Omega \\  			\underline n \cdot \nabla w &= 0 \quad \mathrm{on} \quad \partial \Omega \,, 		\end{split}\end{displaymath}%
\lthtmldisplayZ
\lthtmlcheckvsize\clearpage}

{\newpage\clearpage
\lthtmlinlinemathA{tex2html_wrap_inline7164}%
$ \varphi = (\lambda + c)\,w$%
\lthtmlinlinemathZ
\lthtmlcheckvsize\clearpage}

{\newpage\clearpage
\lthtmlinlinemathA{tex2html_wrap_inline7166}%
$ \Omega = [-1,1]^2$%
\lthtmlinlinemathZ
\lthtmlcheckvsize\clearpage}

{\newpage\clearpage
\lthtmlinlinemathA{tex2html_wrap_inline7168}%
$ \lambda = 2\pi^2$%
\lthtmlinlinemathZ
\lthtmlcheckvsize\clearpage}

{\newpage\clearpage
\lthtmlinlinemathA{tex2html_wrap_inline7170}%
$ w = \cos\pi x\cos\pi y$%
\lthtmlinlinemathZ
\lthtmlcheckvsize\clearpage}

{\newpage\clearpage
\lthtmlinlinemathA{tex2html_wrap_indisplay7172}%
$\displaystyle \underline f = (2\pi^2 + c)\pi \left(\begin{array}{cc} \cos\pi x\sin\pi y \\  -\sin\pi x\cos\pi y \end{array}\right)\,, \qquad
\underline E = \pi \left(\begin{array}{cc} \cos\pi x\sin\pi y \\  -\sin\pi x\cos\pi y \end{array}\right)\,.
$%
\lthtmlindisplaymathZ
\lthtmlcheckvsize\clearpage}

{\newpage\clearpage
\lthtmlinlinemathA{tex2html_wrap_inline7179}%
$ w(x,y) = (1-x^2)^2(1-y^2)^2$%
\lthtmlinlinemathZ
\lthtmlcheckvsize\clearpage}

{\newpage\clearpage
\lthtmlinlinemathA{tex2html_wrap_inline7181}%
$ \underline n \cdot \nabla w = 0$%
\lthtmlinlinemathZ
\lthtmlcheckvsize\clearpage}

{\newpage\clearpage
\lthtmlinlinemathA{tex2html_wrap_inline7183}%
$ \partial [-1,1]^2$%
\lthtmlinlinemathZ
\lthtmlcheckvsize\clearpage}

{\newpage\clearpage
\lthtmlinlinemathA{tex2html_wrap_inline7185}%
$ \varphi = -\Delta w + c w$%
\lthtmlinlinemathZ
\lthtmlcheckvsize\clearpage}

\appendix
\stepcounter{section}
{\newpage\clearpage
\lthtmldisplayA{displaymath7188}%
\begin{displaymath}\begin{split} 	\varepsilon \frac{\partial \mathcal{E}}{\partial t} & = \mathop{\rm curl}\mathcal{H} - \sigma \mathcal{E} \,, \\  	\mu \frac{\partial \mathcal{H}}{\partial t} & = -\mathop{\rm curl}\mathcal{E}\,, \end{split}\end{displaymath}%
\lthtmldisplayZ
\lthtmlcheckvsize\clearpage}

{\newpage\clearpage
\lthtmlinlinemathA{tex2html_wrap_inline7190}%
$ \mathcal{E}$%
\lthtmlinlinemathZ
\lthtmlcheckvsize\clearpage}

{\newpage\clearpage
\lthtmlinlinemathA{tex2html_wrap_inline7192}%
$ \mathcal{H}$%
\lthtmlinlinemathZ
\lthtmlcheckvsize\clearpage}

{\newpage\clearpage
\lthtmlinlinemathA{tex2html_wrap_inline7194}%
$ \varepsilon (x), \mu(x)$%
\lthtmlinlinemathZ
\lthtmlcheckvsize\clearpage}

{\newpage\clearpage
\lthtmlinlinemathA{tex2html_wrap_inline7196}%
$ \sigma(x)$%
\lthtmlinlinemathZ
\lthtmlcheckvsize\clearpage}

{\newpage\clearpage
\lthtmlinlinemathA{tex2html_wrap_inline7202}%
$ L^{\infty}(\Omega )^{d\times d}$%
\lthtmlinlinemathZ
\lthtmlcheckvsize\clearpage}

{\newpage\clearpage
\lthtmlinlinemathA{tex2html_wrap_inline7204}%
$ \varepsilon (x)$%
\lthtmlinlinemathZ
\lthtmlcheckvsize\clearpage}

{\newpage\clearpage
\lthtmlinlinemathA{tex2html_wrap_inline7206}%
$ \mu(x)$%
\lthtmlinlinemathZ
\lthtmlcheckvsize\clearpage}

{\newpage\clearpage
\lthtmlinlinemathA{tex2html_wrap_inline7211}%
$ \mathcal{E}(x,t)$%
\lthtmlinlinemathZ
\lthtmlcheckvsize\clearpage}

{\newpage\clearpage
\lthtmlinlinemathA{tex2html_wrap_inline7213}%
$ \mathcal{H}(x,t)$%
\lthtmlinlinemathZ
\lthtmlcheckvsize\clearpage}

{\newpage\clearpage
\lthtmldisplayA{displaymath7215}%
\begin{displaymath}\begin{split} 	\mathcal{E}(x,t) &= \mathrm{Re} \left(E(x) \exp(i\omega t)\right) \,, \\  	\mathcal{H}(x,t) &= \mathrm{Re} \left(H(x) \exp(i\omega t)\right) \,. \end{split}\end{displaymath}%
\lthtmldisplayZ
\lthtmlcheckvsize\clearpage}

{\newpage\clearpage
\lthtmlinlinemathA{tex2html_wrap_inline7217}%
$ E(x), H(x)$%
\lthtmlinlinemathZ
\lthtmlcheckvsize\clearpage}

{\newpage\clearpage
\lthtmlinlinemathA{tex2html_wrap_inline7219}%
$ \omega\neq 0$%
\lthtmlinlinemathZ
\lthtmlcheckvsize\clearpage}

{\newpage\clearpage
\lthtmlinlinemathA{tex2html_wrap_inline7222}%
$ E(x) \exp(i\omega t)$%
\lthtmlinlinemathZ
\lthtmlcheckvsize\clearpage}

{\newpage\clearpage
\lthtmlinlinemathA{tex2html_wrap_inline7224}%
$ H(x) \exp(i\omega t)$%
\lthtmlinlinemathZ
\lthtmlcheckvsize\clearpage}

{\newpage\clearpage
\lthtmlinlinemathA{tex2html_wrap_inline7226}%
$ H(x)$%
\lthtmlinlinemathZ
\lthtmlcheckvsize\clearpage}

{\newpage\clearpage
\lthtmlinlinemathA{tex2html_wrap_indisplay7228}%
$\displaystyle \mathop{\rm curl}(\mu^{-1} \mathop{\rm curl}E) - \omega^2\varepsilon E + i\omega\sigma E = 0
$%
\lthtmlindisplaymathZ
\lthtmlcheckvsize\clearpage}

{\newpage\clearpage
\lthtmlinlinemathA{tex2html_wrap_inline7230}%
$ |\omega|$%
\lthtmlinlinemathZ
\lthtmlcheckvsize\clearpage}

{\newpage\clearpage
\lthtmlinlinemathA{tex2html_wrap_indisplay7232}%
$\displaystyle \omega^2\varepsilon \ll \mu^{-1} \,,\quad \omega^2\varepsilon \ll \omega\sigma \,.
$%
\lthtmlindisplaymathZ
\lthtmlcheckvsize\clearpage}

{\newpage\clearpage
\lthtmlinlinemathA{tex2html_wrap_inline7234}%
$ \omega^2\varepsilon E(x)$%
\lthtmlinlinemathZ
\lthtmlcheckvsize\clearpage}

{\newpage\clearpage
\lthtmlinlinemathA{tex2html_wrap_indisplay7236}%
$\displaystyle \mathop{\rm curl}(\mu^{-1} \mathop{\rm curl}E) + i\omega\sigma E = 0
$%
\lthtmlindisplaymathZ
\lthtmlcheckvsize\clearpage}

{\newpage\clearpage
\lthtmlinlinemathA{tex2html_wrap_indisplay7242}%
$\displaystyle E \wedge n = \Phi \quad \mathrm{on} \quad \partial \Omega \,.$%
\lthtmlindisplaymathZ
\lthtmlcheckvsize\clearpage}

{\newpage\clearpage
\lthtmlinlinemathA{tex2html_wrap_inline7244}%
$ \tilde{E}$%
\lthtmlinlinemathZ
\lthtmlcheckvsize\clearpage}

{\newpage\clearpage
\lthtmlinlinemathA{tex2html_wrap_inline7246}%
$ \underline u = E - \tilde{E}$%
\lthtmlinlinemathZ
\lthtmlcheckvsize\clearpage}

{\newpage\clearpage
\lthtmldisplayA{displaymath7248}%
\begin{displaymath}\begin{split} 	\mathop{\rm curl}(\mu^{-1} \mathop{\rm curl}u) + i\omega\sigma u &= F \quad \mathrm{in} \quad \Omega \,, \\  	u \wedge n &= 0 \quad \mathrm{on} \quad \partial \Omega \,. \end{split}\end{displaymath}%
\lthtmldisplayZ
\lthtmlcheckvsize\clearpage}


\end{document}
